%%%%%%%%%%%%%
%% Results %%
%%%%%%%%%%%%%

\chapter{Results} \label{Ch:Results}
In this chapter, the results for the experiment described in section~\ref{Sec:Experiments} are reported. This includes ablation studies, parameter tests and registration performance assessment on the \emph{ACDC} dataset as well as a motion-compensated reconstruction pipeline on the \emph{CMRxRecon} dataset.
 
\section{Ablation Studies and Parameter Tests} \label{Sec:ResultsParameterTestsACDC}
In this section, the results for the ablation studies and parameter tests described in section~\ref{SubSec:ParameterTestsACDC} are presented.

\subsection{Fourier-Net versus Fourier-Net+} \label{SubSec:ResultsFourier-NetvsFourier-Net+ACDC}
First, \emph{Fourier-Net}, \emph{Fourier-Net+} and \emph{4xFourier-Net} were compared as described in section~\ref{SubSubSec:Fourier-NetvsFourier-Net+}. The results for the dense displacements can be seen in Table~\ref{tab:Fourier-NetvsFourier-Net+ACDCDense} and for the band-limited displacements in Table~\ref{tab:Fourier-NetvsFourier-Net+ACDCBand-Limited}. For the dense displacement variants, the diffeomorphic transform increases the Dice score (with and without the background label) slightly for \emph{Fourier-Net} and \emph{4xFourier-Net}, but decreases for \emph{Fourier-Net+}. The SSIM decreases for all models slightly, while the MSE remains unaffected. The percentage of non-positive Jacobian determinants is lower for the diffeomorphic variants. Times for the baseline and diffeomorphic versions are similar with the \emph{Fourier-Net} being faster without the diffeomorphism by 1.13~ms, while \emph{Fourier-Net+} and \emph{4xFourier-Net} are faster with the diffeomorphism by 3.28~ms and 1.56~ms respectively. \\
The results for the band-limited versions are very similar with \emph{Fourier-Net} and \emph{Fourier-Net+} having a lower Dice (with background) with the diffeomorphism, while \emph{4xFourier-Net} increases by almost 2$\%$. When excluding the background label \emph{Fourier-Net} is again slightly worse, but \emph{Fourier-Net+} and \emph{4xFourier-Net} both improve, the latter with almost 3$\%$ Dice more than the baseline version. The SSIM values are slightly lower for \emph{Fourier-Net} and \emph{Fourier-Net+} with the diffeomorphism, but again higher for \emph{4xFourier-Net}. The MSE is slightly lower for \emph{Fourier-Net} and \emph{4xFourier-Net}, but slightly higher for \emph{Fourier-Net+}. The percentage of non-positive Jabobian determinants again decreases for the diffeomorphic variants. The times again vary slightly with \emph{Fourier-Net} and \emph{Fourier-Net+} being slower by 0.5~ms and 6.26~ms, while \emph{4xFourier-Net} is a lot faster with 14.68~ms difference.

\begin{table}[h] %htpb
	%\scriptsize
	\centering
	\caption{Results for the \emph{Fourier-Net versus Fourier-Net+} experiment using dense displacements and optional diffeomorphic transforms on the fully sampled \emph{ACDC} test data.}
	\label{tab:Fourier-NetvsFourier-Net+ACDCDense}
	\begin{tabularx}{\textwidth}{Y Y Y Y Y} 
		\toprule
		\multirow{2}{*}{Metrics} & \multicolumn{3}{c}{Dense Displacement} \\
		\cmidrule(lr){2-4} 
		 & Fourier-Net & Fourier-Net+ & 4xFourier-Net+\\	
		\midrule
		$\%$ Dice $\uparrow$ & $78.22 \pm 13.84$ & $78.43 \pm 13.96$ & \textbf{79.34} $\pm$ \textbf{14.03}\\
		$\%$ Dice* $\uparrow$ & $71.03 \pm 18.87$ & $70.50 \pm 19.17$ & \textbf{72.02} $\pm$ \textbf{18.94} \\
		$\%$ SSIM $\uparrow$ & \textbf{91.92} $\pm$ \textbf{3.32} & $89.09 \pm 4.10$ & $89.65 \pm 3.85$\\
		MSE (m) $\downarrow$ & \textbf{0.09} $\pm$ \textbf{0.06} & $0.17 \pm 0.13$ & $0.15 \pm 0.12$ \\
		$\% \, |J_{\phi}|\leq0 \downarrow$ & $0.53 \pm 0.52$ & $0.32 \pm 0.55$ & \textbf{0.04} $\pm$ \textbf{0.09} \\
		Time [ms] $\downarrow$  & \textbf{7.03} 	& 10.74 	& 21.89 \\
		\midrule
		 & \mbox{Diff-Fourier-Net} & \mbox{Diff-Fourier-Net+} & \mbox{Diff-4xFourier-Net+}\\		
		\midrule
		$\%$ Dice $\uparrow$ & $78.56 \pm 14.13$ & $77.88 \pm 13.81$ & \textbf{79.49} $\pm$ \textbf{14.26}\\
		$\%$ Dice* $\uparrow$ & $71.31 \pm 19.13$ & $70.19 \pm 18.82$ & \textbf{72.12} $\pm$ \textbf{19.33} \\
		$\%$ SSIM $\uparrow$ & \textbf{91.79} $\pm$ \textbf{3.38} & $89.08 \pm 4.10$ & $89.62 \pm 3.98$\\
		MSE (m) $\downarrow$ & \textbf{0.09} $\pm$ \textbf{0.06} & $0.16 \pm 0.13$ & $0.15 \pm 0.12$ \\
		$\% \, |J_{\phi}|\leq0 \downarrow$ & $0.03 \pm 0.06$ & $0.00 \pm 0.00$ & $0.00 \pm 0.00$ \\
		Time [ms] $\downarrow$  & 8.16  & \textbf{7.46} & 20.33 \\
		\midrule
		Parameters $\downarrow$ & 645,216 & \textbf{380,470} & 1,521,880 \\
		Mult-Adds (G) $\downarrow$ & 1.12  	& \textbf{0.04} 	& 0.18 \\
		Memory [MB] $\downarrow$ & 97.63   & \textbf{5.81} 	& 21.90 \\
		\bottomrule
	\end{tabularx}
\end{table}

\noindent Now to the difference between the dense and band-limited displacements. Overall, the dense displacement versions have a higher Dice score for all models both with and without the background label.  The dense displacements also have better SSIM and MSE metrics, however the difference is not quite as large as with the Dice scores.  Only the percentage of non-positive Jacobian determinants is notably lower for the band-limited displacement. In terms of inference time, the dense displacement variants are slightly faster despite having a larger encoder, however all of the models are in the milliseconds (7~ms up to 33.5~ms). Last but not least, the memory consumption needs to be addressed. The dense displacment variants have a larger encoder, as discussed before, and thus have far more parameters, especially for the more optimized \emph{Fourier-Net+} and \emph{4xFourier-Net} (about 5 times more). The number of Mult-Adds and the amount of total memory are more than doubled (almost tripled for \emph{4xFourier-Net}) for the dense displacement versions compared to those with a band-limited displacement across all models.

\begin{table}[h]		
	\centering
	\caption{Results for the \emph{Fourier-Net versus Fourier-Net+} experiment using band-limited displacements and optional diffeomorphic transforms on the fully sampled \emph{ACDC} test data.}
	\label{tab:Fourier-NetvsFourier-Net+ACDCBand-Limited}
	\begin{tabularx}{\textwidth}{Y Y Y Y Y}
		\toprule	
		\multirow{2}{*}{Metrics} & \multicolumn{3}{c}{Band-limited Displacement} \\
		\cmidrule(lr){2-4} 
		 & Fourier-Net & Fourier-Net+ & 4xFourier-Net+\\		
		\midrule
		$\%$ Dice $\uparrow$ & \textbf{77.95} $\pm$ \textbf{14.71} & $75.35 \pm 14.28$ & $76.59 \pm 14.84$\\
		$\%$ Dice* $\uparrow$ & \textbf{69.96} $\pm$ \textbf{21.55} & $64.74 \pm 21.12$ & $66.82 \pm 21.55$ \\
		$\%$ SSIM $\uparrow$ & \textbf{91.35} $\pm$ \textbf{3.51} & $88.42 \pm 3.94$ & $88.59 \pm 4.23$\\
		MSE (m) $\downarrow$ & \textbf{1.00} $\pm$ \textbf{0.71} & $2.06 \pm 1.54$ & $1.92 \pm 1.46$ \\
		$\% \, |J_{\phi}|\leq0 \downarrow$ & $0.36 \pm 0.38$ & $0.13 \pm 0.35$ & \textbf{0.03} $\pm$ \textbf{0.12} \\
		Time [ms] $\downarrow$ & \textbf{14.46} & \textbf{14.46} & 33.49 \\
		\midrule
		 & \mbox{Diff-Fourier-Net} & \mbox{Diff-Fourier-Net+} & \mbox{Diff-4xFourier-Net+}\\		
		\midrule
		$\%$ Dice $\uparrow$ & $77.93 \pm 14.93$ & $75.17 \pm 13.43$ & \textbf{78.20} $\pm$ \textbf{14.01}\\
		$\%$ Dice* $\uparrow$ & \textbf{69.91} $\pm$ \textbf{21.99} & $65.67 \pm 18.61$ & $69.61 \pm 19.11$ \\
		$\%$ SSIM $\uparrow$ & \textbf{91.24} $\pm$ \textbf{3.56} & $87.81 \pm 3.94$ & $88.81 \pm 4.13$\\
		MSE (m) $\downarrow$ & \textbf{0.98} $\pm$ \textbf{0.67} & $2.29 \pm 1.61$ & $1.85 \pm 1.41$ \\
		$\% \, |J_{\phi}|\leq0 \downarrow$ & $0.01 \pm 0.03$ & \textbf{0.00} $\pm$ \textbf{0.00} & \textbf{0.00} $\pm$ \textbf{0.00} \\
		Time [ms] $\downarrow$ 	  & \textbf{14.96}  & 20.72 & 18.81 \\
		\midrule
		Parameters $\downarrow$    & 434,519 & \textbf{75,429} & 301,716 \\
		Mult-Adds (M) $\downarrow$ & 595.04  & \textbf{10.98}  & 43.91 \\
		Memory [MB] $\downarrow$   & 44.69   & \textbf{2.25}   & 7.66 \\
		\bottomrule
	\end{tabularx}
\end{table}


\subsection{Starting Channel Size} \label{SubSec:ResultsStartingChannelsACDC}
Next, the impact of the starting channel size was examined as explained in section~\ref{SubSubSec:StartingChannelsACDC}. The results can be seen in Table~\ref{tab:StartingChannelsFourierNet+ACDC}. The Dice (with and without the background label) increased for both \emph{Fourier-Net+} and \emph{4xFourier-Net+} for larger channel sizes. A marginal increase in SSIM and MSE was observed, while the percentage of non-positive Jabobian determinants decreased. The inference time is not heavily impacted by channel size as the times remain very similar for all sizes. Overall, \emph{4xFourier-Net+} is slower than \emph{Fourier-Net+} by roughly a factor of 3 to 4. The number of parameters drastically increases with starting size leading to an increase in Mult-Add and thus overall memory. As \emph{4xFourier-Net+} is just a cascaded version of \emph{Fourier-Net+} a channel size of 16 for the latter is about the same as channel size~8 for the cascaded version and so on. This can also be seen in the direct comparison between \emph{Fourier-Net+} and \emph{4xFourier-Net+}, where the latter has a better performance, but also needs more memory. While an increase of channel size also increases performance in terms of Dice (with the background label) by $0.36\%$, $0.43\%$ and $0.41\%$ for the latter, which is quite consistent, it can be observed that an increase of channel size for \emph{Fourier-Net+} has an bigger impact at the beginning ($1.38\%$, $0.78\%$ and $0.15\%$) and fades towards larger channels sizes. This effect can also be seen when excluding the background label ($0.69\%$, $0.78\%$ and $0.42\%$ for \emph{4xFourier-Net+} compared to $2.16\%$, $1.09\%$ and $0.10\%$ for \emph{Fourier-Net+}), but is less pronounced for the SSIM and MSE. Note that all versions of \emph{Fourier-Net+} and \emph{4xFourier-Net+} are smaller in terms of total memory than \emph{Fourier-Net} with channel size~8 (44.69 MB), except for \emph{4xFourier-Net+} with channel size~64.

\begin{table}[h] %tpb
	%\scriptsize
	\centering
	\caption{Results for the different starting channel sizes of \emph{Fourier-Net+} and \emph{4xFourier-Net+} on the fully sampled \emph{ACDC} test data.}
	\label{tab:StartingChannelsFourierNet+ACDC}
	\begin{tabularx}{\textwidth}{Y Y Y Y Y}
		\toprule
		\multirow{2}{*}{Metrics} & \multicolumn{4}{c}{Starting Channels - Fourier-Net+} \\
		\cmidrule(lr){2-5}
		 & 8 & 16 & 32 & 64 \\		
		\midrule
		$\%$ Dice $\uparrow$ & $75.50 \pm 13.79$ & $76.88 \pm 13.86$ & $77.66 \pm 13.60$ & \mbox{\textbf{77.81} $\pm$ \textbf{13.76}} \\
		$\%$ Dice* $\uparrow$ & $65.70 \pm 18.96$ & $67.86 \pm 19.06$ & $68.95 \pm 18.65$ & \mbox{\textbf{69.05} $\pm$ \textbf{18.80}} \\
		$\%$ SSIM $\uparrow$ & $88.05 \pm 3.89$ & $88.67 \pm 3.88$ & $88.83 \pm 3.83$ & \textbf{89.02} $\pm$ \textbf{3.67} \\
		MSE (m) $\downarrow$ & $0.22 \pm 0.16$ & $0.20 \pm 0.15$ & \textbf{0.19} $\pm$ \textbf{0.15} & \textbf{0.19} $\pm$ \textbf{0.15} \\
		$\% \, |J_{\phi}|\leq0 \downarrow$ & $0.07 \pm 0.25$ & $0.04 \pm 0.14$ & $0.01 \pm 0.04$ & \textbf{0.00} $\pm$ \textbf{0.03} \\
		Time [ms] $\downarrow$ 	  & \textbf{7.18} & 7.77 & 7.99 & 7.79 \\
		Parameters  $\downarrow$	  & \textbf{75,429} 	& 300,477 	& 1,199,469 	& 4,793,037 \\
		\mbox{Mult-Adds (M) $\downarrow$}  & \textbf{10.98} 	& 42.89 		& 169.54 	& 674.10 \\
		\mbox{Memory [MB] $\downarrow$} 	  & \textbf{2.25} 	& 4.64 		& 11.22 		& 31.57 \\
		\midrule		
		\multirow{2}{*}{Metrics} & \multicolumn{4}{c}{Starting Channels - 4xFourier-Net+} \\
		\cmidrule(lr){2-5} 
		 & 8 & 16 & 32 & 64 \\		
		\midrule
		$\%$ Dice $\uparrow$ & $77.54 \pm 13.73$ & $77.90 \pm 13.92$ & $78.33 \pm 14.14$ & \mbox{\textbf{78.74} $\pm$ \textbf{13.91}} \\
		$\%$ Dice* $\uparrow$ & $68.52 \pm 18.62$ & $69.21 \pm 19.02$ & $69.99 \pm 19.21$ & \mbox{\textbf{70.41} $\pm$ \textbf{19.02}} \\
		$\%$ SSIM $\uparrow$ & $88.70 \pm 4.17$ & $88.89 \pm 4.05$ & $89.08 \pm 4.01$ & \textbf{89.29} $\pm$ \textbf{3.74} \\
		MSE (m) $\downarrow$ & $0.19 \pm 0.14$ & $0.19 \pm 0.14$ & \textbf{0.18} $\pm$ \textbf{0.14} & \textbf{0.18} $\pm$ \textbf{0.14} \\
		$\% \, |J_{\phi}|\leq0 \downarrow$ & $0.02 \pm 0.07$ & $0.01 \pm 0.04$ & $0.01 \pm 0.05$ & \textbf{0.00} $\pm$ \textbf{0.00} \\
		Time [ms] $\downarrow$ 	  & 30.13 	& \textbf{24.61} 	& 29.65 	& 27.73 \\
		Parameters $\downarrow$ 	  & \textbf{301,716} 	& 1,201,908 	& 4,797,876 	& 19,172,148 \\
		\mbox{Mult-Adds (G) $\downarrow$} & \textbf{0.04} 	& 0.17 		& 0.68 		& 2.70 \\
		\mbox{Memory [MB]  $\downarrow$}	  & \textbf{7.66} 	& 17.23 		& 43.56 		& 124.94 \\
		\bottomrule
	\end{tabularx}
\end{table}


\subsection{Fourier-Transform Crop Size} \label{SubSec:ResultsFTCropSize}
The results for the FT crop size experiment, as described in section~\ref{SubSubSec:FTCropSize}, can be seen in Table~\ref{tab:FTCropSize}. The Dice score, both with and without the background label, decreases drastically with a smaller crop size. The SSIM and MSE values also get worse (SSIM decreases, MSE increases), however, the percentage of non-positive Jacobian determinants decreases for smaller crop sizes. It should be noted that the crop size of $48 \times 48$ is an outlier in this regard breaking the trend with a higher percentage as $40 \times 84$. The inference time also decreases slightly with a smaller crop size, although this is quite noisy as the time from $80 \times 168$ to $40 \times 84$ is halved for \emph{Fourier-Net+}, but the time needed for $48 \times 48$ and $24 \times 24$ increases again slightly. For \emph{4xFourier-Net+} the time decrease for all smaller crop sizes, except for $24 \times 24$. The number of parameters does not change for different crop sizes as the networks themselves do not change, however the number of Mult-Adds and the total memory still change with the image size. Both decrease with a larger crop size as the image gets smaller. This effect is again not linear as a reduction in image size yields a larger reduction in memory going from $80 \times 168$ to $40 \times 84$ than from $48 \times 48$ to $24 \times 24$. 

\begin{table}[h] %tpb
	%\scriptsize
	\centering
	\caption{Results for four different FT crop sizes for \emph{Fourier-Net+} and \emph{4xFourier-Net+} examined on the fully sampled \emph{ACDC} test data.}
	\label{tab:FTCropSize}
	\begin{tabularx}{\textwidth}{Y Y Y Y Y} 
		\toprule
		\multirow{2}{*}{Metrics} & \multicolumn{4}{c}{FT crop size - Fourier-Net+} \\
		\cmidrule(lr){2-5}
		 & $80 \times 168$ & $40 \times 84$ & $48 \times 48$ & $24 \times 24$ \\		
		\midrule
		$\%$ Dice $\uparrow$ & \mbox{\textbf{78.24} $\pm$ \textbf{14.43}} & $76.61 \pm 13.90$ & $75.27 \pm 13.49$ & $73.77 \pm 14.42$ \\
		$\%$ Dice* $\uparrow$ & \mbox{\textbf{70.66} $\pm$ \textbf{19.70}} & $67.60 \pm 19.24$ & $65.48 \pm 18.76$ & $63.37 \pm 20.07$ \\
		$\%$ SSIM $\uparrow$ & \textbf{89.81} $\pm$ \textbf{4.09} & $88.58 \pm 3.87$ & $87.98 \pm 3.91$ & $87.00 \pm 3.99$ \\
		MSE (m) $\downarrow$ & \textbf{0.15} $\pm$ \textbf{0.12} & $0.20 \pm 0.15$ & $0.22 \pm 0.16$ & $0.27 \pm 0.19$ \\
		$\% \, |J_{\phi}|\leq0 \downarrow$ & $0.22 \pm 0.46$ & $0.05 \pm 0.15$ & $0.09 \pm 0.25$ & \textbf{0.00} $\pm$ \textbf{0.02} \\		
		Time [ms]  $\downarrow$	  & \textbf{6.99} & 7.71 & 7.18 & 8.14 \\
		Parameters  $\downarrow$	  & 75,429 & 75,429 & 75,429 & 75,429 \\
		\mbox{Mult-Adds (M) $\downarrow$} & 64.03  & 16.37  & 10.98  & \textbf{2.74} \\
		\mbox{Memory [MB]  $\downarrow$} 	  & 9.51   & 2.97   & 2.25   & \textbf{1.12} \\
		\midrule
		\multirow{2}{*}{Metrics} & \multicolumn{4}{c}{FT crop size - 4xFourier-Net+} \\
		\cmidrule(lr){2-5}
		 & $80 \times 168$ & $40 \times 84$ & $48 \times 48$ & $24 \times 24$ \\		
		\midrule
		$\%$ Dice $\uparrow$ & \mbox{\textbf{78.12} $\pm$ \textbf{15.01}} & $77.49 \pm 14.67$ & $74.95 \pm 14.15$ & $72.58 \pm 14.67$ \\
		$\%$ Dice* $\uparrow$ & \mbox{\textbf{70.08} $\pm$ \textbf{21.92}} & $68.03 \pm 21.57$ & $64.54 \pm 20.75$ & $61.19 \pm 21.63$ \\
		$\%$ SSIM $\uparrow$ & \textbf{89.91} $\pm$ \textbf{4.01} & $89.03 \pm 3.98$ & $87.98 \pm 3.94$ & $87.02 \pm 3.95$ \\
		MSE (m) $\downarrow$ & \textbf{1.45} $\pm$ \textbf{1.11} & $1.81 \pm 1.37$ & $2.22 \pm 1.61$ & $2.64 \pm 1.84$ \\
		$\% \, |J_{\phi}|\leq0 \downarrow$ & $0.12 \pm 0.23$ & $0.03 \pm 0.15$ & $0.06 \pm 0.23$ & \textbf{0.02} $\pm$ \textbf{0.15} \\	
		Time [ms] $\downarrow$ 	 & 39.03  & 30.05  & 30.13  & \textbf{28.86} \\
		Parameters  $\downarrow$	 & 301,716 & 301,716 & 301,716 & 301,716 \\
		\mbox{Mult-Adds (M) $\downarrow$} & 256.14  & 65.50   & 43.91   & \textbf{10.98} \\
		\mbox{Memory [MB]  $\downarrow$} 	 & 36.70   & 10.54   & 7.66    & \textbf{3.15} \\
		\bottomrule
	\end{tabularx}
\end{table}

\subsection{Comparison with VoxelMorph} \label{SubSec:ResultsComparisonVoxelMorph}
Next is the comparison with \emph{VoxelMorph}, as described in section~\ref{SubSubSec:ComparisonVoxelMorph}. The results can be seen in Table~\ref{tab:CompareVoxelMorph}. \emph{VoxelMorph} has the lowest Dice score with the background label, however it does have a higher value than \emph{Fourier-Net+} for the Dice without the background label. \emph{Fourier-Net} has the highest Dice scores (both with and without the background label), followed by \emph{4xFourier-Net+}. \emph{VoxelMorph} has the best SSIM and MSE values, followed by \emph{Fourier-Net}. The percentage of non-positive Jacobian determinants for \emph{VoxelMorph} is high with $1\%$ while the other models are all under $0.25\%$. In terms of time, \emph{4xFourier-Net+} is slowest followed by \emph{VoxelMorph}, \emph{Fourier-Net+} and \emph{Fourier-Net}, although all models are very fast (less than 25~ms per image pair). The model with the least amount of parameters and Mult-Adds is \emph{VoxelMorph}, but it still requires quite a lot of memory with almost 40~MB. \emph{Fourier-Net} is the largest model in terms of model parameters, Mult-Adds and total memory (90~MB). \emph{Fourier-Net+} and \emph{4xFourier-Net+} both have a higher number of parameters and Mult-Adds, but a lower total amount of memory is needed (5~MB and 17~MB). 

\begin{table}[h] %tpb
	%\scriptsize
	\centering
	\caption{Comparison of \emph{Fourier-Net}, \emph{Fourier-Net+}, \emph{4xFourier-Net+} and \emph{VoxelMorph} with similarity metrics and memory consumption on the fully sampled \emph{ACDC} test data.}
	\label{tab:CompareVoxelMorph}
	
	\begin{tabularx}{\textwidth}{Y Y Y Y Y} 
		\toprule
		 & Fourier-Net & Fourier-Net+ & \mbox{4xFourier-Net+} & VoxelMorph \\		
		\midrule
		$\%$ Dice $\uparrow$ & \mbox{\textbf{78.31} $\pm$ \textbf{13.96}} & $76.88 \pm 13.86$ & $77.90 \pm 13.92$ & $75.84 \pm 13.46$ \\
		$\%$ Dice* $\uparrow$ & \mbox{\textbf{71.17} $\pm$ \textbf{18.99}} & $67.86 \pm 19.06$ & $69.21 \pm 19.02$ & $68.25 \pm 18.36$ \\
		$\%$ SSIM $\uparrow$ & $91.53 \pm 3.49$ & $88.67 \pm 3.88$ & $88.89 \pm 4.05$ & \textbf{93.53} $\pm$ \textbf{3.30} \\
		MSE (m) $\downarrow$ & $0.09 \pm 0.07$ & $0.20 \pm 0.15$ & $0.19 \pm 0.14$ & \textbf{0.06} $\pm$ \textbf{0.04} \\
		$\% \, |J_{\phi}|\leq0 \downarrow$ & $0.24 \pm 0.45$ & $0.02 \pm 0.14$ & \textbf{0.00} $\pm$ \textbf{0.04} & $1.00 \pm 0.57$ \\
		Time [ms] $\downarrow$ 	  & \textbf{7.58}    & 7.78 	& 24.61  	& 10.82 \\
		Parameters $\downarrow$ 	  & 1,735,447 & 300,477 	& 1,201,908 	& \textbf{84,322} \\
		\mbox{Mult-Adds (G) $\downarrow$} & 2.35      & 0.04289  & 0.17157  	& \textbf{0.00157} \\
		\mbox{Memory [MB]  $\downarrow$} 	  & 90.08     & \textbf{4.64}   	& 17.23    	& 39.04 \\
		\bottomrule
	\end{tabularx}	
\end{table}


\subsection{Dense Displacement on Accelerated Data} \label{SubSec:ResultsDenseDisplacementAcc}
The difference between a dense displacement field and a band-limited one was already explored in section~\ref{SubSec:ResultsFourier-NetvsFourier-Net+ACDC}. However, the results in Tables~\ref{tab:Fourier-NetvsFourier-Net+ACDCDense} and~\ref{tab:Fourier-NetvsFourier-Net+ACDCBand-Limited} only include fully sampled, not accelerated, data. These new tests again included \emph{Fourier-Net}, \emph{Fourier-Net+} and \emph{4xFourier-Net+}, but are extended to subsampled data. Results for $R=4$ can be seen in Table~\ref{tab:DenseDisplacementAcc4}, $R=8$ in Table~\ref{tab:DenseDisplacementAcc8} and $R=10$ in Table~\ref{tab:DenseDisplacementAcc10}. \\
For $R=4$, the dense versions of \emph{Fourier-Net}, \emph{Fourier-Net+} and \emph{4xFourier-Net+} have higher Dice scores than the band-limited versions, though the differences are less than a percent with and less than two percent without the background label. The same is true for the SSIM and MSE values with the dense versions being better. It should be noted that \emph{Fourier-Net} performs best for all the aforementioned metrics for the dense and band-limited displacements with the exception of the Dice with the background label where the band-limited \emph{4xFourier-Net+} is better. The percentage of non-positive Jacobian determinants is higher for the band-limited \emph{Fourier-Net}, while band-limited \emph{Fourier-Net+} and \emph{4xFourier-Net+} show almost no folding. Band-limited \emph{Fourier-Net} and \emph{Fourier-Net+} are faster (1.6~ms and 1.76~ms), however, band-limited \emph{4xFourier-Net+} is slower than its dense counterpart by 4.93~ms. Overall, \emph{Fourier-Net+} is the fastest network for both dense and band-limited displacements.

\begin{table}[h] %tpb
	%\scriptsize
	\centering
	\caption{Results for \emph{Fourier-Net}, \emph{Fourier-Net+} and \emph{4xFourier-Net+} with both dense and band-limited displacement fields on the $R=4$ \emph{ACDC} test data.}
	\label{tab:DenseDisplacementAcc4}
	\begin{tabularx}{\textwidth}{Y Y Y Y} 
		\toprule
		\multirow{2}{*}{Metrics} & \multicolumn{3}{c}{Dense Displacement} \\
		\cmidrule(lr){2-4} 
		 & Fourier-Net & Fourier-Net+ & 4xFourier-Net+\\	
		\midrule
		$\%$ Dice $\uparrow$ & \textbf{77.37} $\pm$ \textbf{13.92} & $76.90 \pm 14.12$ & $77.35 \pm 14.04$\\
		$\%$ Dice* $\uparrow$ & \textbf{68.98} $\pm$ \textbf{19.10} & $68.39 \pm 19.30$ & $68.76 \pm 19.32$ \\
		$\%$ SSIM $\uparrow$ & \textbf{84.75} $\pm$ \textbf{7.01} & $79.79 \pm 9.76$ & $80.32 \pm 9.37$\\
		MSE (m) $\downarrow$ & \textbf{0.08} $\pm$ \textbf{0.05} & $0.15 \pm 0.12$ & $0.13 \pm 0.10$ \\
		$\% \, |J_{\phi}|\leq0 \downarrow$ & $0.16 \pm 0.19$ & $0.11 \pm 0.21$ & \textbf{0.03} $\pm$ \textbf{0.07} \\
		Time [ms] $\downarrow$ 	  & 8.49 & \textbf{8.26} & 26.33  \\
		\midrule
		\multirow{2}{*}{Metrics} & \multicolumn{3}{c}{Band-limited Displacement} \\
		\cmidrule(lr){2-4} 
		 & Fourier-Net & Fourier-Net+ & 4xFourier-Net+\\		
		\midrule
		$\%$ Dice $\uparrow$ & $76.55 \pm 13.89$ & $76.20 \pm 13.57$ & \textbf{77.23} $\pm$ \textbf{13.73}\\
		$\%$ Dice* $\uparrow$ & \textbf{67.90} $\pm$ \textbf{19.44} & $66.30 \pm 19.01$ & $67.67 \pm 18.97$ \\
		$\%$ SSIM $\uparrow$ & \textbf{82.93} $\pm$ \textbf{8.12} & $77.74 \pm 10.69$ & $77.96 \pm 10.57$\\
		MSE (m) $\downarrow$ & \textbf{0.10} $\pm$ \textbf{0.06} & $0.19 \pm 0.14$ & $0.18 \pm 0.13$ \\
		$\% \, |J_{\phi}|\leq0 \downarrow$ & $0.35 \pm 0.41$ & $0.01 \pm 0.05$ & \textbf{0.00} $\pm$ \textbf{0.02} \\
		Time [ms] $\downarrow$ 	  & 6.89  	& \textbf{6.50} 	& 31.26  \\
		\bottomrule
	\end{tabularx}
\end{table}

\begin{table}[H] %tpb
	%\scriptsize
	\centering
	\caption{Results for \emph{Fourier-Net}, \emph{Fourier-Net+} and \emph{4xFourier-Net+} with both dense and band-limited displacement fields on the $R=8$ \emph{ACDC} test data.}
	\label{tab:DenseDisplacementAcc8}
	\begin{tabularx}{\textwidth}{Y Y Y Y} 
		\toprule
		\multirow{2}{*}{Metrics} & \multicolumn{3}{c}{Dense Displacement} \\
		\cmidrule(lr){2-4} 
		 & Fourier-Net & Fourier-Net+ & 4xFourier-Net+\\	
		\midrule
		$\%$ Dice $\uparrow$ & $77.48 \pm 14.00$ & $77.30 \pm 13.72$ & \textbf{77.80} $\pm$ \textbf{14.02}\\
		$\%$ Dice* $\uparrow$ & $68.86 \pm 19.34$ & $68.39 \pm 19.15$ & \textbf{69.18} $\pm$ \textbf{19.37} \\
		$\%$ SSIM $\uparrow$ & \textbf{90.08} $\pm$ \textbf{3.05} & $87.60 \pm 3.78$ & $87.91 \pm 3.64$\\
		MSE (m) $\downarrow$ & \textbf{0.05} $\pm$ \textbf{0.04} & $0.10 \pm 0.10$ & $0.09 \pm 0.09$ \\
		$\% \, |J_{\phi}|\leq0 \downarrow$ & $0.11 \pm 0.15$ & $0.06 \pm 0.13$ & \textbf{0.02} $\pm$ \textbf{0.06} \\
		Time [ms] $\downarrow$ 	  & 11.46 & \textbf{10.79} & 38.55  \\
		\midrule
		\multirow{2}{*}{Metrics} & \multicolumn{3}{c}{Band-limited Displacement} \\
		\cmidrule(lr){2-4} 
		 & Fourier-Net & Fourier-Net+ & 4xFourier-Net+\\		
		\midrule
		$\%$ Dice $\uparrow$ & $75.52 \pm 13.69$ & $75.76 \pm 13.43$ & \textbf{76.56} $\pm$ \textbf{13.47}\\
		$\%$ Dice* $\uparrow$ & \textbf{66.86} $\pm$ \textbf{18.97} & $65.45 \pm 18.63$ & $66.81 \pm 18.77$ \\
		$\%$ SSIM $\uparrow$ & \textbf{89.51} $\pm$ \textbf{3.34} & $86.30 \pm 4.18$ & $86.29 \pm 4.23$\\
		MSE (m) $\downarrow$ & \textbf{0.06} $\pm$ \textbf{0.04} & $0.13 \pm 0.11$ & $0.13 \pm 0.11$ \\
		$\% \, |J_{\phi}|\leq0  \downarrow$ & $0.27 \pm 0.29$ & $0.03 \pm 0.11$ & \textbf{0.00} $\pm$ \textbf{0.02} \\
		Time [ms] $\downarrow$ 	  & 8.97 & \textbf{7.71} & 46.58  \\
		\bottomrule
	\end{tabularx}
\end{table}

\noindent For $R=8$, all band-limited networks have a lower Dice and SSIM values with a higher MSE. Band-limited \emph{4xFourier-Net+} again has the highest Dice with the background for the band-limited networks while \emph{Fourier-Net} has the highest Dice without background, SSIM and lowest MSE. While it also previously performed best for all these metrics for the dense displacements, \emph{4xFourier-Net+} now has the highest Dice scores for all dense networks. The percentage of non-positive Jacobian determinants is again higher for the band-limited \emph{Fourier-Net} compared to the dense version, while the band-limited \emph{Fourier-Net+} and \emph{4xFourier-Net+} show almost no folding. Band-limited \emph{Fourier-Net} and \emph{Fourier-Net+} are faster by 2.49~ms and 3.08~ms while \emph{4xFourier-Net+} is slower compared to its dense counterpart by 8.03~ms. For both dense and band-limited networks, \emph{Fourier-Net+} is repeatedly the fastest network.

\begin{table}[h] %tpb
	%\scriptsize
	\centering
	\caption{Results for \emph{Fourier-Net}, \emph{Fourier-Net+} and \emph{4xFourier-Net+} with both dense and band-limited displacement fields on the $R=10$ \emph{ACDC} test data.}
	\label{tab:DenseDisplacementAcc10}
	\begin{tabularx}{\textwidth}{Y Y Y Y} 
		\toprule
		\multirow{2}{*}{Metrics} & \multicolumn{3}{c}{Dense Displacement} \\
		\cmidrule(lr){2-4} 
		 & Fourier-Net & Fourier-Net+ & 4xFourier-Net+\\	
		\midrule
		$\%$ Dice $\uparrow$ & $77.39 \pm 13.89$ & $77.31 \pm 13.82$ & \textbf{77.43} $\pm$ \textbf{13.99}\\
		$\%$ Dice* $\uparrow$ & $68.86 \pm 19.27$ & $68.69 \pm 19.07$ & \textbf{68.90} $\pm$ \textbf{19.43} \\
		$\%$ SSIM $\uparrow$ & \textbf{92.92} $\pm$ \textbf{2.68} & $91.09 \pm 3.42$ & $91.33 \pm 3.29$\\
		MSE (m) $\downarrow$ & \textbf{0.05} $\pm$ \textbf{0.04} & $0.09 \pm 0.09$ & $0.08 \pm 0.09$ \\
		$\% \, |J_{\phi}|\leq0 \downarrow$ & $0.11 \pm 0.15$ & $0.06 \pm 0.11$ & \textbf{0.04} $\pm$ \textbf{0.11} \\
		Time [ms] $\downarrow$ 	  & 27.96 & \textbf{9.49} & 18.31  \\
		\midrule
		\multirow{2}{*}{Metrics} & \multicolumn{3}{c}{Band-limited Displacement} \\
		\cmidrule(lr){2-4} 
		 & Fourier-Net & Fourier-Net+ & 4xFourier-Net+\\		
		\midrule
		$\%$ Dice $\uparrow$ & $76.87 \pm 13.97$ & $76.10 \pm 13.83$ & \textbf{76.98} $\pm$ \textbf{13.57}\\
		$\%$ Dice* $\uparrow$ & \textbf{68.24} $\pm$ \textbf{19.27} & $66.27 \pm 19.20$ & $67.40 \pm 18.90$ \\
		$\%$ SSIM $\uparrow$ & \textbf{92.77} $\pm$ \textbf{2.74} & $90.32 \pm 3.29$ & $90.42 \pm 3.33$\\
		MSE (m) $\downarrow$ & \textbf{0.05} $\pm$ \textbf{0.04} & $0.12 \pm 0.12$ & $0.11 \pm 0.11$ \\
		$\% \, |J_{\phi}|\leq0 \downarrow$ & $0.11 \pm 0.15$ & $0.01 \pm 0.05$ & \textbf{0.00} $\pm$ \textbf{0.03} \\
		Time [ms] $\downarrow$ 	  & \textbf{8.06} & 9.62 & 27.89  \\
		\bottomrule
	\end{tabularx}
\end{table}
 

\noindent For $R=10$, the dense networks have higher Dice and SSIM values compared to the band-limited versions. For the MSE, band-limited \emph{Fourier-Net} has similar values to its dense counterpart, while band-limited \emph{Fourier-Net+} and \emph{4xFourier-Net+} are slightly higher compared to the dense versions. For the dense versions \emph{4xFourier-Net+} has the highest Dice scores, while \emph{Fourier-Net} has the highest SSIM value and lowest MSE value. Band-limited \emph{4xFourier-Net+} has the highest Dice score with the background with band-limited \emph{Fourier-Net} has the highest Dice score without background, SSIM and lowest MSE. The percentage of non-positive Jacobian determinants for the band-limited \emph{Fourier-Net} is equal to the percentage of the dense version, while band-limited \emph{Fourier-Net+} and \emph{4xFourier-Net+} again show almost no folding. Band-limited \emph{Fourier-Net} is faster with 19.9~ms, although the time for dense network is probably an outlier while the band-limited version is fastest for the band-limited networks. \emph{Fourier-Net+} and \emph{4xFourier-Net+} are slower compared to their dense versions with 0.13~ms and 9.58~ms. For the dense networks, \emph{Fourier-Net+} is the fastest network.

 
\section{Registration Performance on Subsampled Data} \label{SubSec:ResultsComparisonSubsampling}
In section~\ref{SubSec:ResultsComparisonVoxelMorph}, \emph{Fourier-Net}, \emph{Fourier-Net+} and \emph{4xFourier-Net+} were already compared to \emph{VoxelMorph}, however, these comparisons were only done on fully sampled, not accelerated data. To further add to the comparison, \emph{NiftyReg} was used as a traditional registration algorithm. The unaligned test image pairs were treated as a baseline for the lower bound of registration performance. The results can be seen in Table~\ref{tab:ComparisonSubsamplingACDC} for fully sampled ($R=0$) and subsampled ($R=4$, $R=8$, $R=10$) \emph{ACDC} test data. Values which are worse than the baseline are marked in red, while the best results for each subsampling and metric is highlighted with blue. All times are computed on CPU due to \emph{NiftyReg} only working on the CPU as the GPU capable versions were deprecated. Additionally, Figure~\ref{fig:Boxplots_DiceScores} shows the performance of the methods for each segmentation label (excluding the background) while Figure~\ref{fig:TestExamples} shows example images and segmentations warped by the different methods for a visual comparison.\\
For $R=0$, \emph{Fourier-Net} has the highest Dice score (both with and without the background label), closely followed by \emph{4xFourier-Net+}, while \emph{NiftyReg} has a lower score than the baseline. \emph{VoxelMorph} has the highest SSIM, followed by \emph{Fourier-Net} and \emph{NiftyReg}, as well as the lowest MSE where \emph{NiftyReg} has a higher value than the baseline. \emph{NiftyReg} and \emph{4xFourier-Net+} have the lowest percentage of non-positive Jacobian determinants while \emph{VoxelMorph} has the highest value with $1\%$, thus indicating potential folding. \emph{Fourier-Net} is the fastest method with under $0.1s$ per image pair, while \emph{NiftyReg} needs over $100s$ on the fully sampled data, making it the slowest by a large margin (all other methods were under $1s$).\\
For $R=4$, the registration performance decreases for all methods with \emph{NiftyReg} having lower Dice scores than the baseline, while \emph{4xFourier-Net+} has the highest Dice score with the background label and \emph{Fourier-Net} the highest Dice score without the background. \emph{VoxelMorph} has the highest SSIM and lowest MSE values, followed by \emph{Fourier-Net} and \emph{NiftyReg}, with none of methods performing worse than the baseline. \emph{4xFourier-Net+} again has the lowest percentage of non-positive Jacobian determinants, closely followed by \emph{Fourier-Net+}, while \emph{VoxelMorph}, similar to the fully sampled data, has the highest value with $1\%$. In terms of time, \emph{NiftyReg} is again the slowest method with about $80s$, while all other methods need around $0.1s$ with \emph{Fourier-Net+} being the fastest.\\
For $R=8$, \emph{4xFourier-Net+} has the highest Dice score with the background label, while \emph{Fourier-Net} has the highest Dice score without the background label. \emph{NiftyReg} has a lower Dice than the baseline (both with and without the background label). \emph{VoxelMorph} again performs best for SSIM and MSE followed by \emph{Fourier-Net} and \emph{NiftyReg}. \emph{4xFourier-Net+} has the lowest percentage of non-positive Jacobian determinants followed by \emph{Fourier-Net+}, while \emph{VoxelMorph} again has the highest value. \emph{NiftyReg} is the slowest method, as before, with over $80s$, similar to $R=4$, while \emph{Fourier-Net+} is again the fastest despite being slightly slower than before.\\
For $R=10$, \emph{4xFourier-Net+} has the highest Dice score with the background label, while \emph{Fourier-Net} has the highest Dice score without the background. \emph{NiftyReg} repeatedly has a lower Dice score than the baseline (both with and without background label), while \emph{VoxelMorph} is the best method in terms of SSIM and MSE, followed by \emph{Fourier-Net} and \emph{NiftyReg} with no method being worse than the baseline for these metrics. \emph{4xFourier-Net+} has the lowest percentage of non-positive Jacobian determinants closely followed by \emph{Fourier-Net+}, while \emph{VoxelMorph} has the highest value like before. \emph{Fourier-Net+} is the fastest method with under $0.01s$, while \emph{NiftyReg} is again the slowest despite its best time yet with about $47s$.


\begin{table}[h] %tpb
	%\tiny
	%\scriptsize
	\footnotesize
	\centering
	\caption{Test results for \emph{NiftiReg} (NR), \emph{VoxelMorph} (VM), \emph{Fourier-Net} (F-Net), \emph{Fourier-Net+} (F-Net+) and \emph{4xFourier-Net+} (4xF-Net+) on the \emph{ACDC} test data from fully sampled ($R=0$) to $R=10$ with an unaligned baseline for comparison. The best results for each metric and subsampling are highlighted in blue, while values worse than the unaligned baseline are marked with red. Arrows indicate whether a lower or higher value is better.}
	\label{tab:ComparisonSubsamplingACDC}
	\resizebox{\textwidth}{!}{
	\begin{tabular}{c c c c c c c c} 
		\toprule
		  & \multirow{2}{*}{Methods} & \multicolumn{6}{c}{Metrics} \\
		\cmidrule(lr){3-8} 
		  & & $\%$ DICE $\uparrow$ & $\%$ DICE* $\uparrow$ & $\%$ SSIM $\uparrow$ & MSE (m) $\downarrow$ & $\% \, |J_{\phi}|\leq0 \downarrow$ & Time [s] $\downarrow$ \\
		% Fully Sampled (R=0)
		\midrule
		\multirow{6}{*}{\rotatebox{90}{$R=0$}} & Baseline & $70.85 \pm 18.27$  & $60.35 \pm 25.24$ & $86.39 \pm 4.08$ & $0.33 \pm 0.23$ & - & -\\  
		 & NR & \textcolor{red}{$70.74 \pm 13.77$} & \textcolor{red}{$57.56 \pm 18.86$} & $91.26 \pm 3.18$ & \textcolor{red}{$1.94 \pm 1.61$} & \textcolor{blue}{$0.00 \pm 0.02$} & 122.52\\
		 & VM & $75.84 \pm 13.46$ & $68.25 \pm 18.36$ & \textcolor{blue}{$93.53 \pm 3.30$} & \textcolor{blue}{$0.06 \pm 0.04$} & $1.00 \pm 0.57$ & 0.1845\\ 
		 & F-Net & \textcolor{blue}{$78.31 \pm 13.96$} & \textcolor{blue}{$71.17 \pm 18.99$} & $91.53 \pm 3.49$ & $0.09 \pm 0.07$ & $0.24 \pm 0.25$ & 0.1918\\ 
		 & F-Net+ & $76.88 \pm 13.86$ & $67.86 \pm 19.06$ & $88.67 \pm 3.88$ & $0.20 \pm 0.15$ & $0.02 \pm 0.08$ & \textcolor{blue}{0.0893} \\ 
		 & 4xF-Net+ & $77.90 \pm 13.92$ & $69.21 \pm 19.02$  & $88.89 \pm 4.05$ & $0.19 \pm 0.14$ & \textcolor{blue}{$0.00 \pm 0.02$} & 0.3262\\ 
		 	
		% 4x Accelerated (R=4) 				 		
		\midrule
		\multirow{6}{*}{\rotatebox{90}{$R=4$}} & Baseline & $70.85 \pm 18.27$ & $60.35 \pm 25.24$ & $76.80 \pm 11.02$ & $0.28 \pm 0.19$ & - & -\\  
		 & NR & \textcolor{red}{$69.89 \pm 13.73$} & \textcolor{red}{$56.47 \pm 17.86$} & $86.03 \pm 6.41$ & $0.15 \pm 0.14$ & $ 0.12 \pm 0.15$ & 80.08 \\  
		 & VM & $71.78 \pm 14.57$ & $63.15 \pm 18.98$ & \textcolor{blue}{$90.73 \pm 4.72$} & \textcolor{blue}{$0.04 \pm 0.03$} & $1.00 \pm 1.10$ & 0.1264\\  	
		 & F-Net & $76.55 \pm 13.89$ & \textcolor{blue}{$67.90 \pm 19.44$} & $82.93 \pm 8.12$ & $0.10 \pm 0.06$ & $0.19 \pm 0.23$ & 0.1006\\ 
		 & F-Net+ & $76.20 \pm 13.57$ & $66.30 \pm 19.01$ & $77.74 \pm 10.69$ & $0.19 \pm 0.14$ & $0.01 \pm 0.03$ & \textcolor{blue}{0.0294}\\ 
		 & 4xF-Net+ & \textcolor{blue}{$77.23 \pm 13.73$} & $67.67 \pm 18.97$ & $77.96 \pm 10.57$ & $0.18 \pm 0.13$ & \textcolor{blue}{$0.00 \pm 0.02$} & 0.1131\\   
		
		% 8x Accelerated (R=8) 
		\midrule
		\multirow{6}{*}{\rotatebox{90}{$R=8$}} & Baseline & $70.85 \pm 18.27$ & $60.35 \pm 25.24$ & $85.35 \pm 4.43$ & $0.22 \pm 0.17$ & - & -\\  
		 & NR & \textcolor{red}{$70.04 \pm 13.42$} & \textcolor{red}{$56.37 \pm 17.63$} & $91.07 \pm 2.80$ & $0.12 \pm 0.13$ & $0.08 \pm 0.10$ & 88.36 \\
		 & VM & $71.51 \pm 14.22$ & $62.17 \pm 18.80$ & \textcolor{blue}{$94.17 \pm 2.80$} & \textcolor{blue}{$0.03 \pm 0.02$} & $1.00 \pm 0.96$ & 0.1973\\	
		 & F-Net & $75.52 \pm 13.69$ & \textcolor{blue}{$66.86 \pm 18.97$} & $89.51 \pm 3.34$ & $0.06 \pm 0.04$ & $0.27 \pm 0.29$ & 0.2404\\ 
		 & F-Net+ & $75.76 \pm 13.43$ & $65.45 \pm 18.63$ & $86.30 \pm 4.18$ & $0.13 \pm 0.11$ & $0.03 \pm 0.11$ & \textcolor{blue}{0.1482}\\ 
		 & 4xF-Net+ & \textcolor{blue}{$76.56 \pm 13.47$} & $66.81 \pm 18.77$ & $86.29 \pm 4.23$ & $0.13 \pm 0.11$ & \textcolor{blue}{$0.00 \pm 0.02$} & 0.5283\\ 
		 	 
		% 10x Accelerated (R=10) 		 		
		\midrule		
		\multirow{6}{*}{\rotatebox{90}{$R=10$}} & Baseline & $70.85 \pm 18.27$ & $60.35 \pm 25.24$ & $89.17 \pm 3.58$ & $0.20 \pm 0.17$ & - & -\\ 
		 & NR & \textcolor{red}{$70.40 \pm 13.34$} & \textcolor{red}{$56.61 \pm 17.71$} & $93.47 \pm 2.37$ & $0.11 \pm 0.13$ & $0.06 \pm 0.08$ & 47.44 \\
		 & VM & $71.89 \pm 13.94$ & $62.22 \pm 18.58$ & \textcolor{blue}{$95.85 \pm 2.58$} & \textcolor{blue}{$0.02 \pm 0.02$} & $1.00 \pm 0.96$ & 0.0577\\	 %0.2120
		 & F-Net & $76.87 \pm 13.97$ & \textcolor{blue}{$68.24 \pm 19.27$} & $92.77 \pm 2.74$ & $0.05 \pm 0.04$ & $0.11 \pm 0.15$ & 0.0296\\ 
		 & F-Net+ & $76.10 \pm 13.83$ & $66.27 \pm 19.20$ & $90.32 \pm 3.29$ & $0.12 \pm 0.12$ & $0.01 \pm 0.05$ & \textcolor{blue}{0.0059}\\ 
		 & 4xF-Net+ & \textcolor{blue}{$76.98 \pm 13.57$} & $67.40 \pm 18.90$ & $90.42 \pm 3.33$ & $0.11 \pm 0.11$ & \textcolor{blue}{$0.00 \pm 0.03$} & 0.0275\\ 
		 \bottomrule
	\end{tabular}}
\end{table}


\subsubsection{Cardiac Labels}
To further evaluate the difference between the methods, one can look at the Dice scores of the three individual cardiac labels. These can be seen as boxplots in Figure~\ref{fig:Boxplot_DiceScores_LV-Cavity} for the left ventricle (LV), Figure~\ref{fig:Boxplot_DiceScores_Myocardium} for the myocardium and Figure~\ref{fig:Boxplot_DiceScores_RV-Cavity} for the right ventricle (RV). The Dice scores for each label vary widely but the performance of all methods is best for the RV and worst for the myocardium. This seems to be an underlying principle of the data as the same behavior can be seen in the unaligned baseline. \\
\emph{NiftyReg} has high scores for the LV, even surpassing \emph{VoxelMorph}, while it has the lowest values for all methods on the myocardium (slightly higher than the baseline) and the RV (far lower than the baseline). \\
\emph{VoxelMorph} has the highest scores for the RV compared to the other labels being close the \emph{Fourier-Net} variant for $R=0$, but falls off for the subsampled data, even below the baseline. It surpasses the baseline, \emph{NiftyReg} and even \emph{Fourier-Net+} on $R=0$ for the myocardium, but again loses performance for the subsampled data. This trend also continues for the LV were \emph{VoxelMorph} has values similar to the baseline, but higher than \emph{NiftyReg}, but ends up falling below the baseline again. \\
\emph{Fourier-Net} consistently has the highest values for the challenging myocardium, but for the LV and RV it has similar scores as \emph{Fourier-Net+} and \emph{4xFourier-Net+}, even having lower values than the baseline for $R=8$ on the latter. \\
\emph{Fourier-Net+} performs well for the LV despite the subsampling, although it is surpassed by \emph{Fourier-Net} on $R=0$ and by \emph{4xFourier-Net+} for $R=4$ and $R=10$. For the challenging myocardium \emph{Fourier-Net+} again has good scores, but is surpassed by \emph{Fourier-Net} for all subsamplings and only manages to surpass \emph{4xFourier-Net+} for $R=10$. For the RV \emph{Fourier-Net+} is among the best methods, having higher values than \emph{Fourier-Net} and \emph{4xFourier-Net+} for $R=8$.\\
\emph{4xFourier-Net+} consistently has the best median value for the LV and is close to having the highest score for the RV. Only on the myocardium it has consistently lower values than \emph{Fourier-Net}.


\begin{figure}[H]
	\vspace{-.5cm}
	\centering
	\graphicspath{{images/}{\main/images/}}
	\begin{subfigure}{0.83\textwidth}
    		\includegraphics[width=\textwidth]{Boxplot_DiceScores_LV-Cavity.pdf}
    		\caption{Boxplot of the Dice scores for the LV cavity.} %on the fully sampled \emph{ACDC} test data
    		\label{fig:Boxplot_DiceScores_LV-Cavity}
	\end{subfigure}
	\\
	\begin{subfigure}{0.83\textwidth}
    		\includegraphics[width=\textwidth]{Boxplot_DiceScores_Myocardium.pdf}
    		\caption{Boxplot of the Dice scores for the myocardium.} %on the Acc4 \emph{ACDC} test data
    		\label{fig:Boxplot_DiceScores_Myocardium}
	\end{subfigure}
	\\
	\begin{subfigure}{0.83\textwidth}
    		\includegraphics[width=\textwidth]{Boxplot_DiceScores_RV-Cavity.pdf}
    		\caption{Boxplot of the Dice scores for RV cavity.} %on the Acc8 \emph{ACDC} test data
    		\label{fig:Boxplot_DiceScores_RV-Cavity}
	\end{subfigure}
	\caption{Boxplots of Dice scores (split by label excluding the background) for all models on fully sampled ($R=0$) and subsampled ($R=4$, $R=8$, $R=10$) \emph{ACDC} test data. \legendsquare{DicePurple}~Baseline, \legendsquare{DiceBlue}~\emph{NiftyReg}, \legendsquare{DiceTeal}~\emph{VoxelMorph}, \legendsquare{DiceYellow}~\emph{Fourier-Net}, \legendsquare{DiceOrange}~\emph{Fourier-Net+}, \legendsquare{DiceRed}~\emph{4xFourier-Net+}.}
	\label{fig:Boxplots_DiceScores}
\end{figure}


\subsubsection{Qualitative Results}
In order to explain the differences observed between the methods previously, a visual examination can be used. There, one can look at the warped images and segmentations as well as the generated displacements visualized in Figure~\ref{fig:TestExamples}. For all subsamplings, an example moving and fixed image can be compared to the warped images by the different methods. For further analysis, the corresponding segmentations (with Dice scores) and displacement fields generated by the methods are given.\\
For $R=0$, \emph{Fourier-Net+} and \emph{4xFourier-Net+} have very localized displacements, centered on the cardiac region, leading to a very smooth warped segmentation compared to \emph{Fourier-Net} and \emph{VoxelMorph} where the cardiac labels mix in some regions. Despite this, \emph{Fourier-Net} still achieves the highest Dice score with an $16\%$ increase compared to the baseline. While \emph{NiftyReg} does produce a very smooth warped segmentation the actual displacements are very small and not localized on the cardiac region leading to the lowest Dice score of all methods.\\
For $R=4$, the image artifacts due to the subsampling are very apparent. The segmentations for the moving and fixed image obviously remain the same. \emph{Fourier-Net+} and \emph{4xFourier-Net+} again have very smooth segmentations, however the displacements reflect the difficulties of adapting to subsampled data in becoming slightly less local. Surprisingly, \emph{NiftyReg} actually has a more localized displacement compared to the fully sampled data, however the segmentation looks less smooth and the Dice score is slightly lower. Somewhat similar, \emph{VoxelMorph} also has a more localized displacement leading to a smoother segmentation and a higher Dice score. The results of \emph{Fourier-Net} look overall very similar to the fully sampled case, perhaps with a bit more localized displacement leading to a smoother segmentation, although this does not lead to a higher Dice score.\\
For $R=8$, the displacements for all methods become more global due to the strong presence of image artifacts, however the Dice scores are higher than before for all methods. \emph{NiftyReg} even outperforms \emph{VoxelMorph} for the first time. This trend does not continue for $R=10$ however, were \emph{NiftyReg} is again far below all other methods in terms of Dice with a very much more global displacement. \emph{VoxelMorph} seems to compensate more for the rippling artifacts present in the fixed image than for the change in the cardiac region as seen in the displacement. \emph{Fourier-Net}, \emph{Fourier-Net+} and \emph{4xFourier-Net+} do not share this behavior although their displacements also become more global and less focused on the cardiac region. The latter network has the highest Dice score for the first time managing an increase in Dice of about $16\%$ compared to the baseline for this specific case despite the heavy subsampling showing the robustness of the network.

\begin{figure}[H]
	\vspace{-1cm}
	\centering
	\graphicspath{{images/}{\main/images/}}
	\includegraphics[width=.95\textwidth]{TestExamples_Mode0.pdf}
    	\includegraphics[width=.95\textwidth]{TestExamples_Mode1.pdf}
    	\includegraphics[width=.95\textwidth]{TestExamples_Mode2.pdf}
    	\includegraphics[width=.95\textwidth]{TestExamples_Mode3.pdf}	
	\caption{\small Examples of warped images, segmentations and flow fields for \emph{NiftiReg}, \emph{VoxelMorph}, \emph{Fourier-Net}, \emph{Fourier-Net+} and \emph{4xFourier-Net+} together with the original image pair from the fully sampled ($R=0$) and subsampled ($R=4$, $R=8$, $R=10$) \emph{ACDC} test data.}
	\label{fig:TestExamples}
\end{figure}


\section{Integration into a Motion-Compensated Reconstruction Pipeline} \label{Sec:ResultsIntegrationMotion-CompensatedReconstructionPipeline}
After assessing the registration performance of \emph{Fourier-Net}, \emph{Fourier-Net+} and \emph{4xFourier-Net+}, a final downstream task to gauge the applicability of these networks was conducted in the form of an motion-compensated reconstruction pipeline where the networks are used to correct the movement between frames.

\subsection{K-Space Line Swapping} \label{SubSubSec:ResultsK-SpaceLineSwapping}
As discussed in section~\ref{SubSec:IntegrationMotion-CompensatedReconstructionPipeline}, the reconstruction pipeline was first tested with motion-corrupted data using k-space line swapping described in section~\ref{SubSec:SimulatedMotion}. The results are shown in Table~\ref{tab:ComparisonReconstructionCMRxReconLineSwapping} with blue marking the best results per metric (not present if all methods have the same performance) and red marking worse results than the baseline. \\
For $z=16$, performance decreases with higher acceleration. For $R=4$, \emph{Fourier-Net} has the highest HaarPSI and SSIM values, while \emph{VoxelMorph} has the highest PSNR. \emph{Fourier-Net+} on the other hand, has the lowest values for these metrics. All methods as well as the baseline have the same mean MSE value. Only for the third decimal point in the standard deviation a small difference between the different methods is visible (note that the MSE values are already multiplied by a factor of 100). For $R=8$, \emph{Fourier-Net} again has the highest value for the HaarPSI, followed by \emph{Fourier-Net+}, as well as the highest SSIM value, followed by \emph{4xFourier-Net+} which also has the highest PSNR. While the MSE values are again very close, \emph{4xFourier-Net+} has a slightly lower mean value than the other methods. For $R=10$, \emph{4xFourier-Net+} performs best for all metrics with \emph{Fourier-Net+} having the same mean MSE value. \emph{VoxelMorph} and \emph{Fourier-Net} perform worse than baseline for the HaarPSI with the latter also having lower PSNR and SSIM values than the baseline.\\
For $z=32$ and $R=4$, \emph{Fourier-Net} performs best for all metrics, while \emph{4xFourier-Net+} performs worse than baseline for PSNR, SSIM and MSE. For $R=8$, \emph{Fourier-Net+} has the highest HaarPSI, PSNR and SSIM values followed by \emph{Fourier-Net}. The mean MSE value is again the same for all methods and slightly lower than the baseline. For $R=10$, \emph{Fourier-Net+} again performs best for all metrics. Only \emph{Fourier-Net} has a lower SSIM than the baseline.

\begin{table}[H] %tpb
	%\footnotesize
	\small
	\centering
	\caption{Reconstruction results \emph{VoxelMorph}, \emph{Fourier-Net}, \emph{Fourier-Net+} and \emph{4xFourier-Net+} on the \emph{CMRxRecon} test data for $R=4$, $R=8$ and $R=10$ as well as an baseline without motion-correction. The best results for each metric and subsampling are highlighted in blue, while values worse than the unaligned baseline are marked with red.}
	\label{tab:ComparisonReconstructionCMRxReconLineSwapping}
	\begin{tabularx}{\textwidth}{c Y Y Y Y Y} 
		\toprule
		 & \multirow{2}{*}{Methods} & \multicolumn{4}{c}{Motion-correction for $z=16$ swapped k-space lines} \\
		\cmidrule(lr){3-6} 
		 & & $\%$ HaarPSI $\uparrow$ & PSNR [dB] $\uparrow$ & $\%$ SSIM $\uparrow$ & MSE (m) $\downarrow$\\
		
		% 4x Accelerated (R=4) 				 		
		\midrule
		\multirow{5}{*}{\rotatebox{90}{$R=4$}} & Baseline & $56.884 \pm 8.129$ & $28.703 \pm 2.388$ & $78.079 \pm 5.887$ & $0.160 \pm 0.125$ \\  
		 & VoxelMorph & $56.909 \pm 8.118$ & \textcolor{blue}{$28.709 \pm 2.404$} & $78.090 \pm 5.903$ & $0.160 \pm 0.126$ \\  
		 & Fourier-Net & \textcolor{blue}{$56.918 \pm 8.160$} & $28.706 \pm 2.401$ & \textcolor{blue}{$78.123 \pm 5.893$} & $0.160 \pm 0.126$ \\  
		 & Fourier-Net+ & \textcolor{red}{$56.854 \pm 8.158$} & \textcolor{red}{$28.699 \pm 2.399$} & \textcolor{red}{$78.071 \pm 5.951$} & $0.160 \pm 0.126$ \\   
		 & \mbox{4xFourier-Net+} & $56.903 \pm 8.109$ & $28.702 \pm 2.380$ & $78.085 \pm 5.877$ & $0.160 \pm 0.125$ \\  
		
		% 8x Accelerated (R=8) 
		\midrule
		\multirow{5}{*}{\rotatebox{90}{$R=8$}} & Baseline & $53.318 \pm 7.641$ & $28.104 \pm 2.383$ & $77.311 \pm 5.934$ & $0.183 \pm 0.139$ \\  
		 & VoxelMorph & $53.342 \pm 7.687$ & $28.110 \pm 2.405$ & $77.341 \pm 5.900$ & $0.183 \pm 0.139$ \\  
		 & Fourier-Net & \textcolor{blue}{$53.394 \pm 7.681$} & $28.128 \pm 2.398$ & \textcolor{blue}{$77.416 \pm 5.927$} & $0.183 \pm 0.138$ \\  
		 & Fourier-Net+ & $53.388 \pm 7.664$ & $28.118 \pm 2.404$ & $77.398 \pm 5.906$ & $0.183 \pm 0.138$ \\   
		 & \mbox{4xFourier-Net+} & $53.376 \pm 7.688$ & \textcolor{blue}{$28.133 \pm 2.398$} & $77.402 \pm 5.936$ & \textcolor{blue}{$0.182 \pm 0.138$} \\ 
		 	 
		% 10x Accelerated (R=10) 		 		
		\midrule		
		\multirow{5}{*}{\rotatebox{90}{$R=10$}} & Baseline & $52.212 \pm 7.388$ & $27.906 \pm 2.364$ & $77.175 \pm 5.886$ & $0.191 \pm 0.141$ \\  
		 & VoxelMorph & \textcolor{red}{$52.211 \pm 7.393$} & $27.907 \pm 2.368$ & $77.194 \pm 5.863$ & $0.191 \pm 0.140$ \\  
		 & Fourier-Net & \textcolor{red}{$52.192 \pm 7.361$} & \textcolor{red}{$27.895 \pm 2.362$} & \textcolor{red}{$77.160 \pm 5.815$} & $0.191 \pm 0.139$ \\  
		 & Fourier-Net+ & $52.240 \pm 7.379$ & $27.924 \pm 2.357$ & $77.244 \pm 5.866$ & \textcolor{blue}{$0.189 \pm 0.139$} \\   
		 & \mbox{4xFourier-Net+} & \textcolor{blue}{$52.272 \pm 7.330$} & \textcolor{blue}{$27.931 \pm 2.349$} & \textcolor{blue}{$77.262 \pm 5.816$} & \textcolor{blue}{$0.189 \pm 0.137$} \\ 
		 
		 \midrule	
		 & & \multicolumn{4}{c}{Motion-correction for $z=32$ swapped k-space lines} \\
		% 4x Accelerated (R=4) 				 		
		\midrule
		\multirow{5}{*}{\rotatebox{90}{$R=4$}} & Baseline & $52.711 \pm 7.673$ & $27.506 \pm 2.180$ & $74.379 \pm 5.869$ & $0.203 \pm 0.130$ \\  
		 & VoxelMorph & $52.747 \pm 7.658$ & $27.512 \pm 2.188$ & $74.349 \pm 5.900$ & $0.203 \pm 0.129$ \\  
		 & Fourier-Net & \textcolor{blue}{$52.802 \pm 7.708$} & \textcolor{blue}{$27.536 \pm 2.197$} & \textcolor{blue}{$74.421 \pm 5.937$} & \textcolor{blue}{$0.202 \pm 0.128$} \\  
		 & Fourier-Net+ & $52.725 \pm 7.724$ & $27.522 \pm 2.198$ & $74.366 \pm 5.892$ & $0.203 \pm 0.129$ \\   
		 & \mbox{4xFourier-Net+} & $52.711 \pm 7.718$ & \textcolor{red}{$27.500 \pm 2.205$} & \textcolor{red}{$74.353 \pm 5.986$} & \textcolor{red}{$0.204 \pm 0.132$} \\  
		
		% 8x Accelerated (R=8) 
		\midrule
		\multirow{5}{*}{\rotatebox{90}{$R=8$}} & Baseline & $50.071 \pm 7.259$ & $27.153 \pm 2.199$ & $74.033 \pm 5.844$ & $0.221 \pm 0.138$ \\  
		 & VoxelMorph & $50.091 \pm 7.301$ & $27.164 \pm 2.198$ & $74.060 \pm 5.890$ & $0.220 \pm 0.138$ \\  
		 & Fourier-Net & $50.152 \pm 7.325$ & $27.178 \pm 2.202$ & $74.094 \pm 5.826$ & $0.220 \pm 0.140$ \\  
		 & Fourier-Net+ & \textcolor{blue}{$50.159 \pm 7.362$} & \textcolor{blue}{$27.196 \pm 2.221$} & \textcolor{blue}{$74.119 \pm 5.853$} & $0.220 \pm 0.141$ \\   
		 & \mbox{4xFourier-Net+} & $50.128 \pm 7.276$ & $27.165 \pm 2.201$ & $74.054 \pm 5.842$ & $0.220 \pm 0.137$ \\ 
		 	 
		% 10x Accelerated (R=10) 		 		
		\midrule		
		\multirow{5}{*}{\rotatebox{90}{$R=10$}} & Baseline & $49.193 \pm 7.022$ & $27.013 \pm 2.170$ & $73.971 \pm 5.788$ & $0.227 \pm 0.143$ \\  
		 & VoxelMorph & $49.241 \pm 7.047$ & $27.026 \pm 2.179$ & $73.976 \pm 5.795$ & $0.227 \pm 0.142$ \\  
		 & Fourier-Net & $49.255 \pm 7.011$ & $27.024 \pm 2.168$ & \textcolor{red}{$73.952 \pm 5.793$} & $0.227 \pm 0.141$ \\  
		 & Fourier-Net+ & \textcolor{blue}{$49.296 \pm 7.001$} & \textcolor{blue}{$27.045 \pm 2.172$} & \textcolor{blue}{$74.028 \pm 5.784$} & \textcolor{blue}{$0.226 \pm 0.140$} \\   
		 & \mbox{4xFourier-Net+} & $49.287 \pm 7.066$ & $27.032 \pm 2.176$ & $74.022 \pm 5.820$ & $0.227 \pm 0.143$ \\ 
		 \bottomrule
	\end{tabularx}
\end{table}


\subsection{Non-Linear Lung Transformations} \label{SubSubSec:ResultsLungMovement}
As discussed in section~\ref{SubSec:IntegrationMotion-CompensatedReconstructionPipeline}, a second test with simulated non-linear lung motion was conducted. Results are shown in Table~\ref{tab:ComparisonReconstructionCMRxReconLungMovement} with blue marking the best results per metric. For $R=4$, \emph{Fourier-Net} has the highest HaarPSI, PSNR and SSIM value followed by \emph{Fourier-Net+} and \emph{4xFourier-Net+}. \emph{Fourier-Net} also has the lowest MSE value again followed by \emph{Fourier-Net+} and \emph{4xFourier-Net+}. \emph{VoxelMorph} performs better than the baseline, but stays behind \emph{Fourier-Net}, \emph{Fourier-Net+} and \emph{4xFourier-Net+}. For $R=8$, \emph{Fourier-Net+} overtakes \emph{Fourier-Net} for HaarPSI, PSNR, SSIM and MSE followed by \emph{4xFourier-Net+}. \emph{VoxelMorph} is again the weakest method. Overall, the metrics are very much affected by the increased subsampling as the HaarPSI is about $4 \%$ lower for the methods compared to $R=4$. The PSNR, SSIM and MSE are not effected quite as much, but are also lower in the case of PSNR, SSIM by about $1 \%$ and the MSE is higher by about $0.03 \cdot 10^{-2}$. The results for $R=10$ are very similar to $R=8$. \emph{Fourier-Net+} again has the highest HaarPSI, PSNR and SSIM followed by \emph{4xFourier-Net+} and \emph{Fourier-Net}. This time the decrease in metrics from $R=10$ compared to $R=8$ is not quite as large (about $0.5 \%$, $0.2 \%$ and $0.1 \%$ respectively). \emph{Fourier-Net+} also has the lowest MSE again followed by \emph{4xFourier-Net+} similar to $R=8$ as the metrics barely changed.

\begin{table}[h] %tpb
	%\footnotesize
	\small
	\centering
	\caption{Reconstruction results \emph{VoxelMorph}, \emph{Fourier-Net}, \emph{Fourier-Net+} and \emph{4xFourier-Net+} on the \emph{CMRxRecon} test data for $R=4$, $R=8$ and $R=10$ as well as an baseline without motion-compensation. The best results for each metric and subsampling are highlighted in blue, while values worse than the unaligned baseline are marked with red.}
	\label{tab:ComparisonReconstructionCMRxReconLungMovement}
	\begin{tabularx}{\textwidth}{c Y Y Y Y Y} 
		\toprule
		 & \multirow{2}{*}{Methods} & \multicolumn{4}{c}{Metrics} \\
		\cmidrule(lr){3-6} 
		  & & $\%$ HaarPSI $\uparrow$ & PSNR [dB] $\uparrow$ & $\%$ SSIM $\uparrow$ & MSE (m) $\downarrow$\\
		
		% 4x Accelerated (R=4) 				 		
		\midrule
		\multirow{5}{*}{\rotatebox{90}{$R=4$}} & Baseline & $28.882 \pm 3.802$ & $23.291 \pm 1.654$ & $70.375 \pm 4.467$ & $0.504 \pm 0.197$ \\  
		 & VoxelMorph & $54.177 \pm 7.495$ & $28.347 \pm 2.369$ & $79.873 \pm 4.878$ & $0.174 \pm 0.138$ \\ 
		 & Fourier-Net & \textcolor{blue}{$59.886 \pm 8.876$} & \textcolor{blue}{$29.399 \pm 2.766$} & \textcolor{blue}{$82.731 \pm 5.296$} & \textcolor{blue}{$0.141 \pm 0.127$} \\  
		 & Fourier-Net+ & $59.868 \pm 8.824$ & $29.303 \pm 2.761$ & $82.572 \pm 5.037$ & $0.146 \pm 0.142$  \\    
		 & \mbox{4xFourier-Net+} & $59.884 \pm 8.875$ & $29.365 \pm 2.764$ & $82.704 \pm 5.293$ & $0.144 \pm 0.141$ \\ 
		
		% 8x Accelerated (R=8) 
		\midrule
		\multirow{5}{*}{\rotatebox{90}{$R=8$}} & Baseline & $28.572 \pm 3.773$ & $23.173 \pm 1.686$ & $70.603 \pm 4.533$ & $0.519 \pm 0.207$ \\  
		 & VoxelMorph & $50.537 \pm 7.213$ & $27.596 \pm 2.460$ & $78.848 \pm 5.121$ & $0.209 \pm 0.161$ \\ 
		 & Fourier-Net & $55.326 \pm 8.415$ & $28.465 \pm 2.846$ & $81.383 \pm 5.489$ & $0.191 \pm 0.129$ \\  
		 & Fourier-Net+ & \textcolor{blue}{$55.406 \pm 8.432$} & \textcolor{blue}{$28.519 \pm 2.799$} & \textcolor{blue}{$81.447 \pm 5.478$} & \textcolor{blue}{$0.178 \pm 0.162$} \\    
		 & \mbox{4xFourier-Net+} & $55.402 \pm 8.463$ & $28.508 \pm 2.916$ & $81.436 \pm 5.329$ & $0.181 \pm 0.153$ \\ 
		 	 
		% 10x Accelerated (R=10) 		 		
		\midrule		
		\multirow{5}{*}{\rotatebox{90}{$R=10$}} & Baseline & $28.486 \pm 3.766$ & $23.132 \pm 1.695$ & $70.742 \pm 4.436$ & $0.524 \pm 0.211$ \\  
		 & VoxelMorph & $49.072 \pm 6.854$ & $27.331 \pm 2.417$ & $78.756 \pm 4.962$ & $0.220 \pm 0.163$ \\ 
		 & Fourier-Net & $54.012 \pm 8.046$ & $28.217 \pm 2.758$ & $81.224 \pm 5.323$ & $0.193 \pm 0.162$ \\  
		 & Fourier-Net+ & \textcolor{blue}{$54.025 \pm 8.053$} & \textcolor{blue}{$28.228 \pm 2.745$} & \textcolor{blue}{$81.264 \pm 5.331$} & \textcolor{blue}{$0.189 \pm 0.164$} \\    
		 & \mbox{4xFourier-Net+} & $54.016 \pm 8.064$ & $28.223 \pm 2.691$ & $81.236 \pm 5.237$ & $0.191 \pm 0.148$ \\ 
		 \bottomrule
	\end{tabularx}
\end{table}

\subsubsection{Qualitative Results}
Similar to the registration performance on the \emph{ACDC} dataset in section~\ref{SubSec:ResultsComparisonSubsampling}, visual examples can give insights and reasons for trends observed in the metrics. These examples can be seen for \emph{VoxelMorph}, \emph{Fourier-Net}, \emph{Fourier-Net+} and \emph{4xFourier-Net+} in Figure~\ref{fig:ResultsReconstruction}. There, the reconstruction results using the different networks for motion-compensation and the motion-corrupted baseline are compared to the fully sampled ground truth image using error maps for all three subsampling factors $R=4$, $R=8$ and $R=10$.\\
In the top row, the images for the ground truth, the subsampled and motion-corrupted baseline, as well as the reconstructions using \emph{VoxelMorph}, \emph{Fourier-Net}, \emph{Fourier-Net+} and \emph{4xFourier-Net+} are shown. At the top of the images the SSIM values calculated in comparison to the ground truth are shown. It should be noted that all images were croped around the cardiac region for better visualization of the details in the images. Lastly, in the bottom row, the error maps show the differences between the images and the ground truth. The color bar on the left indicates the severity of the deviation going from blue (no/small difference) to yellow (large difference).\\
For $R=4$, as seen before in Table~\ref{tab:ComparisonReconstructionCMRxReconLungMovement}, \emph{Fourier-Net} has the highest SSIM. But what about the visual look? Starting with the motion-corrupted baseline, it is clear that the image is affected by artifacts. The cardiac region looks blurred and distorted with a big tear being visible. This is most likely the result of the motion-corruption as the subsampling only leads to blurring and aliasing artifacts, which are also visible. Continuing with \emph{VoxelMorph}, it is clear that the reconstruction has not been able to restore the image as parts of the cardiac region are still blurred and a small tear can be seen. This can also be seen in the error map where subsampling artifacts can still be observed. Despite the superior performance in terms of metrics, there are still some subsampling artifacts visible for \emph{Fourier-Net}. The cardiac region, however, looks far better compared to \emph{VoxelMorph}, though still slightly blurred when compared to the ground truth. Next up is \emph{Fourier-Net+}, where more artifacts are again present. The cardiac region is again strongly blurred which can also be seen in the error map. \emph{4xFourier-Net+} also shows some blurring but not as much as \emph{Fourier-Net+}. Despite this the SSIM for the image is slightly lower than for \emph{Fourier-Net+}.\\
For $R=8$, \emph{Fourier-Net+} has the highest SSIM followed by \emph{4xFourier-Net+}. \emph{VoxelMorph} again shows a lot of artifacts with a small tear and strong blurring in the cardiac region similar to the baseline. \emph{Fourier-Net} on the other hand has far less artifacts though the whole image still appears quite blurred. The image for \emph{Fourier-Net+} looks less blurred however artifacts from the subsampling are still visible. \emph{4xFourier-Net+} again looks more like \emph{Fourier-Net} before with stronger blurring and failing to remove some of the motion artifacts. However, it still has a slightly better SSIM.\\
For $R=10$, \emph{Fourier-Net+} has the highest SSIM, although the value is lower than for $R=8$. \emph{VoxelMorph} shows artifacts similar to before, though without a tear in the cardiac region this time. \emph{Fourier-Net} also has a lot of artifacts distorting the image similar to \emph{VoxelMorph}. \emph{Fourier-Net+} is mostly free of artifacts except for some blurring. \emph{4xFourier-Net+}, similar to \emph{Fourier-Net}, has large blurring artifacts in the cardiac region and the contrast seems to be slightly different.

\begin{figure}[H] %tpb
	\centering
	\includegraphics[width=\textwidth]{./Images/ResultsReconstruction_mode1.pdf} 
	\includegraphics[width=\textwidth]{./Images/ResultsReconstruction_mode2.pdf}
	\includegraphics[width=\textwidth]{./Images/ResultsReconstruction_mode3.pdf}
	\caption{Visual results of the reconstruction pipeline for \emph{VoxelMorph}, \emph{Fourier-Net}, \emph{Fourier-Net+} and \emph{4xFourier-Net+} as well as the ground truth image and the motion-corrupted baseline on the \emph{CMRxRecon} test data for $R=4$, $R=8$ and $R=10$. Images are cropped around the cardiac region for better visibility of image details.}
	\label{fig:ResultsReconstruction}
\end{figure}