%%%%%%%%%%%%%%%%%%%%%%%%%
%%%%%  Introduction %%%%%
%%%%%%%%%%%%%%%%%%%%%%%%%

\chapter{Introduction} \label{Ch:Introduction}
Magnetic Resonance Imaging (MRI) is a commonly used medical imaging technique based on measuring magnetic fields. 
%It has a lot of benefits because it is non-invasive, radiation-free and has great contrast for soft tissue. However, speed is its main weakness as full-body scan can take up to 30 minutes. This can be a burden on patients due to needing to remain still for a long amount of time in the scanner as well as hindering efficiency in terms of patients per day. 
Despite its many benefits, the method is weighed down by long acquisition times.
Thus, MRI acquisition is usually accelerated by subsampling the k-space in which the raw MR data is recorded. This, however, leads to problematic image artifacts that hinder further processing. Another problem of the slow data acquisition are motion artifacts which are common for organs like lung and heart. 
%While breathing can be controlled for a short amount of time, cardiac motion is involuntarily and should not be stopped. \\
These challenges, which are further discussed in the next section, have been traditionally addressed with computationally intensive algorithms.
% that often iterativly solve an optimization problem. This however, also requires a lot of time and computational resources in post-processing. 
New approaches based on deep neural networks promise to rectify these problems. Neural networks have seen a rise in popularity in recent years in fields such as image processing due to breakthroughs enabled through rapid improvements in computational hardware (e.g. GPUs)~\cite{Chen2020}. While segmentation networks like the \emph{U-Net}~\cite{U-Net} were established quite early with great success, usage of these architectures for registration and even networks for aliasing-free MR reconstruction like~\cite{Kuestner2022,Ghoul2024,Pan2024} have recently come into focus~\cite{Chen2020,Haskins2020}. %\\
In this thesis, the possibility of using an unsupervised deep learning approach to align subsampled MRI data as well its potential usage in a motion reconstruction pipeline will be investigated.
%While MRI is a widely used medical imaging technology, it suffers from long acquisition times that hamper its effectiveness. This is not only uncomfortable for the patients, but also hinders further image processing as the k-space data acquired from the scanner are often subsampled to speed up the process. When reconstructing images from the raw k-space, subsampling leads to artifacts in the image space limiting the amount of acceleration that can be reasonably achieved. Long acquisition times also lead to potential movement between image frames as motion from e.g. the lung and heart can not be suppressed leading to further motion artifacts. All these problems are addressed in this thesis by using an efficient unsupervised neural network for image registration. After evaluating the registration performance to conventional iterative registration algorithms and conventional registration networks, the applicability of the network in a motion-compensated reconstruction pipeline is also tested on artificially motion-corrupted cardiac MR data.

\section{Motivation and Challenges of MRI Acquisition} \label{Sec:MotivationChallengesMRIAcquisition}
MRI has a variety of benefits because it is non-invasive, radiation-free and has great contrast for soft tissue. While it is comparable in many regards to Computed Tomography (CT), it is often favored by physicians due to these reasons~\cite{Liu2013}. However, acquisition speed is its main weakness as a full-body scan can take up to 30 minutes depending on the number of slices scanned~\cite{Brown2014}. This is a burden on patients due to needing to remain still for a long amount of time in the scanner as well as hindering efficiency. In emergency situations, for example, CT will generally be used instead of MRI as the radiation is acceptable when in need of quick diagnostic imaging~\cite{Liu2013}. To understand the nuances of this comparison further, the general process of MR acquisition needs to be understood.\\
There are many challenges with MR acquisition due to the nature of the image technology. Firstly, the images are not acquired directly in image space, but the raw data is instead measured in the so-called k-space. This is a Fourier space holding the image frequencies from which the scans can be reconstructed using an inverse Fourier transform (iFT)~\cite{Brown2014}. The frequencies stem from the magnetic resonance of the measured subject. The magnetic excitation and relaxation, however, takes time and is impossible to speed up. This leads to rather large acquisition times as the process needs to be repeated for each line in the k-space to acquire a full-resolution image. For a spin-echo imaging sequence the acquisition time of $5 \pm 10$ ms for each of the $192 \pm 256$ phase-encoding steps with leads to a total net acquisition time of $1 \pm 2$ s per slice. However, the total acquisition time of the sequence is even longer due to the repetition time defined by e.g. the T1-contrast~\cite{SamplingStrategies}. An easy approach for acceleration is to simply measure less lines and fill the missing lines with zeros~\cite{DeepMRIReconstructionSubsampling}. This is usually done for the higher frequencies as these hold information of edges and finer structures in the image, which are deemed less important than the image contrast stored in lower frequencies. This missing information, however, leads to artifacts upon reconstruction as the Shannon-Nyquist sampling theorem~\cite{Shannon} is violated. Consequently, this kind of acceleration is usually called under- or subsampling of the k-space. The more lines are zeroed out, the worse the artifacts become as more information is missing. This leads to problems for further processing as many algorithms struggle with these artifacts~\cite{DeepMRIReconstructionSubsampling}. \\
Another problem with the long acquisition times is patient movement. While most patient can remain relatively still for up to 30 minutes of measuring, the lung and heart movement cannot simply be stopped. Breathholds can be imployed, however these further slows down the acquisition process and is not deemed a suitable solution. Instead, the motion is often estimated and compensated after reconstruction using registration algorithms or networks~\cite{Kuestner2022, Chen2023,GRICS}.

\section{Related Work} \label{Sec:RelatedWork}
The networks used for the main parts of this thesis are are \emph{Fourier-Net}~\cite{Fourier-Net} and \emph{Fourier-Net+}~\cite{Fourier-Net+}. These are based on works such as the unsupervised \emph{VoxelMorph}~\cite{Voxelmorph}, which is also used for comparison later, as well as the \emph{IC-Net}~\cite{IC-Net} and \emph{SYM-Net}~\cite{SYM-Net}. Additionally, the traditional registration algorithm \emph{NiftyReg}~\cite{NiftiReg} was also used in the evaluation.\\
There are many works on image registration, though the field of medical image registration with deep learning is more limited. A brief overview of registration methods, fundamentals of deep learning with already existing networks for image registration as well as covering potential applications and challenges can be found in~\cite{Chen2020,Haskins2020,Fu2020,Zou2022,Chen2023}. Information on MRI in general, including MRI sampling and acceleration, can be found in~\cite{Serai2021,SamplingStrategies,PulseSequences,AdvancesPI,CS-MRI}.\\
For MRI reconstruction, a lot of work is based on traditional iterative algorithms that are commonly used today~\cite{AdvancesPI,CS-MRI,ParallelMRI,GRAPPA}, however, there exist also deep learning approaches~\cite{DeepMRIReconstructionRadialSubsampling,DeepMRIReconstructionSubsampling}
addressing the problem. While motion in time-series data if often corrected after the reconstruction, an interesting use-case it the motion-compensated reconstruction with can be achieved by combining motion estimation with reconstruction. Both of these parts can be either a traditional algorithm~\cite{GRICS} or neural networks~\cite{Pan2024,Zou2024}.


\section{Contributions and Structure of the Thesis} \label{Sec:ContributionsAndStructure}
Expanding already established work in~\cite{Fourier-Net,Fourier-Net+}, unsupervised neural network training was used to efficiently align subsampled MR images. While the network architectures were already tested for e.g. inter-patient brain scan alignment in the literature, a new use-case with undersampled cardiac MR scan was explored. The registration performance was also compared to both traditional iterative registration algorithms and other neural networks. There, the applicability to this new task was proven with better results compared to the other methods. Furthermore, a potential use-case in a motion-compensated image reconstruction pipeline was tested showing the networks potential. Main benefits compared to traditional algorithms are additionally computation speed and the low memory consumption compared to other neural networks.\\
In chapter~\ref{Ch:Fundamentals}, the foundations of the thesis are introduced. This includes general information about MRI such as basic magnetic principles in section~\ref{SubSec:MagneticExcitationAndRelaxation} image acquisition (section~\ref{SubSec:ImageAcquisitionAndK-Space}), acceleration (section~\ref{SubSec:ImagingAccelerationReconstruction}), and reconstruction (section~\ref{SubSec:Motion-CompensatedReconstruction}). Building of the latter, image transformations (section~\ref{SubSec:ImageTransformations}) and registration (section~\ref{SubSec:ImageRegistration}) are introduced followed by deep learning basics such as common architectures (section~\ref{SubSec:DeepLearningArchitectures}), their use in image registration (section~\ref{SubSec:DLImageRegistration}) and general principles of network training and testing (section~\ref{SubSec:NetworkTrainingAndTesting}).\\
In chapter~\ref{Ch:MasterialsAndMethods}, the materials and methods of the thesis are explained. This includes the specific network architectures in section~\ref{Sec:NetworkArchitecture}, namely \emph{Fourier-Net} (section~\ref{SubSec:Fourier-Net}) and \emph{Fourier-Net+} (section~\ref{SubSec:Fourier-Net+}), followed by the datasets in section\ref{Sec:Datasets}.
% \emph{CMRxRecon} (section~\ref{Sec:CMRxRecon}) and \emph{ACDC} (section~\ref{Sec:ACDC})
Lastly, the conducted experiments are explained in section~\ref{Sec:Experiments}.\\
In chapter~\ref{Ch:Results} the results for these experiments are presented and discussed in chapter~\ref{Ch:Discussion}. This includes a look at limitations and future work followed by a brief summary in chapter~\ref{Ch:Conclusion}.
