%%%%%%%%%%%%%%%%%%%%%%%
%%%%%  Conclusion %%%%%
%%%%%%%%%%%%%%%%%%%%%%%

\chapter{Conclusion} \label{Ch:Conclusion}
In this thesis, the usage of \emph{Fourier-Net}, \emph{Fourier-Net+} and \emph{4xFourier-Net+} for efficient and fast registration of subsampled MRI data was explored. This research question was approached from multiple different angles examining registration both directly, comparing the networks to traditional algorithms and other neural networks, as well as indirectly as part of a motion-compensated reconstruction pipeline. \\
After training the network on subsampled MRI data, multiple experiments were described to examine the research question. These included multiple ablation studies and parameter tests on the \emph{ACDC} dataset followed by downstream tests on the \emph{CMRxRecon} dataset. While the latter was for testing the applicability of the networks, the parameter tests were used to find optimal model parameters for \emph{Fourier-Net}, \emph{Fourier-Net+} and \emph{4xFourier-Net+}. Then their registration performance was compared to \emph{NiftyReg}, a traditional registration algorithm, and \emph{VoxelMorph}, a state-of-the-art neural network.\\
For the second part of experiments, the three networks were used as part of an motion-compensated reconstruction pipeline. For this, the \emph{CMRxRecon} dataset was used as it contained the needed k-space data and subsampling masks. We simulated motion for a downstream test using a motion-compensated reconstruction pipeline. Two different approaches were used to simulate mis-triggering and non-linear lung motion between frames. These were then reconstructed using a pipeline with neural networks for motion-compensation. The three networks were again compared to \emph{VoxelMorph} in this downstream test.