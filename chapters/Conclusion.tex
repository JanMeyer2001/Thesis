%%%%%%%%%%%%%%%%%%%%%%%
%%%%%  Conclusion %%%%%
%%%%%%%%%%%%%%%%%%%%%%%

\chapter{Conclusion} \label{Ch:Conclusion}
In this thesis, the usage of \emph{Fourier-Net}, \emph{Fourier-Net+} and \emph{4xFourier-Net+} for efficient and fast registration of subsampled MRI data was explored. This research question was approached from multiple different angles examining registration both directly, comparing the networks to traditional algorithms as well as other neural networks, as well as indirectly as part of a larger motion-compensated reconstruction pipeline. \\
After explaining the foundations of MR imaging, deep learning and image registration, multiple experiments were described to examine the research question. These included multiple parameter tests and ablation studies on the \emph{ACDC} dataset followed by downstream tests on the \emph{CMRxRecon} dataset. While the latter was for testing the applicability of the networks, the parameter tests were used to find optimal model parameters for \emph{Fourier-Net}, \emph{Fourier-Net+} and \emph{4xFourier-Net+}. Then their registration performance was compared to \emph{NiftyReg}, a traditional registration algorithm, and \emph{VoxelMorph}, a state-of-the-art neural network, on the \emph{ACDC} dataset. The dataset provided segmentations which could be used for accurate measurements of the registration of the images for the data using four different reduction factors. The three networks performed very well for the different amounts of subsampling present in the dataset with \emph{Fourier-Net} generally performing best, though \emph{4xFourier-Net+} had similar results while being more efficient. These finding support the research question, but a practical test was still in order.\\
For the second part of experiments, the three networks were used as part of an motion-compensated reconstruction pipeline. For this, the \emph{CMRxRecon} dataset was used as it contained the needed k-space data and subsampling masks. The data was, however, further corrupted as the motion present in the time series data was not deemed severe enough. Two different approaches were used to simulate mis-triggering and (lung) motion between frames. These were then reconstructed using a pipeline with neural networks for motion-correction. The three networks were again compared to \emph{VoxelMorph} in this downstream test. \emph{Fourier-Net} and \emph{Fourier-Net+} performed better than the other networks in this task and showed that, while there is still a room for improvement, the research question can be answered positively.\\
Some further additions and/or aspects not examined in this thesis are the training of a single network for different reduction factors, changes to the network architecture and/or training for better performance without a larger memory imprint, and evaluation on different data and more diverse tasks.