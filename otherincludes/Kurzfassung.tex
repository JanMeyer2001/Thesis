\chapter*{Abstract}
In this thesis, the usage of specialized neural networks for efficient and fast image registration of subsampled MRI data was explored. This research question was approached from multiple different angles examining registration performance as well as usability in a larger motion-compensated reconstruction pipeline. The foundations of MR imaging, deep learning and image registration are explained and multiple experiments described. These included diverse parameter tests and ablation studies as well as downstream tests on two different datasets to find optimal model parameters as well as testing the applicability of the networks. The registration performance is compared to traditional registration algorithms and state-of-the-art neural networks using segmentations for accurate assessment across four different reduction factors. The networks performed very well both in the tests leading a positive answer to the posed research question. Non the less, some further additions and improvements to the networks and experimental setup are proposed.

\vspace*{1.5cm}
\begin{Huge}
\noindent \textbf{Kurzfassung}
\end{Huge}
\chapterheadstartvskip \\
In dieser Arbeit wird die Nutzung spezialisierter Neuraler Netwerke für effiziente und schnelle Bild-Registrierung von unterabgetasteter MRT-Daten untersucht. Diese Forschungsfrage wurde aus mehreren Blickwinkeln betrachtet und untersucht die Registrierungsqualität sowie Nutzbarkeit in einer Bewegungskompensierten Rekonstruktions-Pipeline. Die Grundlagen von MR Bildgebgung, Deep Learning und Bild-Registrierung werden erklärt und mehrere Experimente beschrieben. Diese beinhalten diverse Parameter Tests und Ablationsstudien sowie Downstream Tests auf zwei verschiedenen Datensätzen um optimale Netzwerk Parameter zu finden und die Anwendbarkeit der Netzwerke zu testen. Die Registrierungsqualität wird mit traditionellen Registierungsalgorithmen und aktuellen Neuralen Netzwerken verglichen  wobei Segmentierungen for eine akkurate Beurteilung über verschiedene Unterabtastungsraten ermöglicht. Die Netzwerke zeigen sehr gute Leistungen in beiden Tests was für eine positive Beantwortung der Forschungsfrage spricht. Nichtsdestotrotz werden weitere Verbesserungen und Erweiterungen von den Netzwerken und Experimenten besprochen.
