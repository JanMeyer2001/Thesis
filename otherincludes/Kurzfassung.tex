\chapter*{Abstract}
While MRI is a widely used medical imaging technology, it suffers from long acquisition times that hamper its effectiveness. This is not only uncomfortable for the patients, but also hinders further image processing as the k-space data acquired from the scanner are often subsampled to speed up the process. When reconstructing images from the raw k-space, subsampling leads to artifacts in the image space limiting the amount of acceleration that can be reasonably achieved. Long acquisition times also lead to potential movement between image frames as motion from e.g. the lung and heart can not be suppressed leading to further motion artifacts. All these problems are addressed in this thesis by using an efficient unsupervised neural network for image registration. After evaluating the registration performance to conventional iterative registration algorithms and conventional registration networks, the applicability of the network in a motion-compensated reconstruction pipeline is also tested on artificially motion-corrupted cardiac MR data.

\vspace*{1.5cm}
\begin{Huge}
\noindent \textbf{Kurzfassung}
\end{Huge}
\chapterheadstartvskip \\
Irgendwas über MRT und Registration
