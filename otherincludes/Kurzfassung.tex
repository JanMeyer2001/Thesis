\chapter*{Abstract}
Magnetic resonance imaging is widely used in the medical field due to its has many benefits, but suffers from long acquisition times that hamper its effectiveness. To combat this, the k-space data acquired from the scanner is accelerated to speed up the process. This, however, leads to artifacts in the reconstructed images proportionate to the amount of subsampling. Long acquisition times also lead to patient discomfort, which can cause the patient to move. Especially physiological motion from e.g. the lung and heart is hard to suppress. The extend of it can be mitigated by breath-holds, however, this is very difficult for sick or sedated patients. These problems can be addressed by image registration which was traditionally done by computationally expensive and slow iterative algorithms. In this thesis, the usage of specialized neural networks for efficient and fast image registration of subsampled magnetic resonance imaging data is explored. The research question is approached from multiple different angles examining registration performance as well as usability in a larger motion-compensated reconstruction pipeline. The foundations of magnetic resonance imaging, deep learning and image registration are explained and multiple experiments described. These include diverse parameter tests and ablation studies as well as downstream tests on two different datasets to find optimal model parameters and test the applicability of the networks. The registration performance is compared to traditional registration algorithms and state-of-the-art neural networks using segmentations for accurate assessment across four different rates of subsampling. We were able to outperform both traditional registration algorithms and state-of-the-art neural networks in terms of Dice score computed on the segmentations even for strongly subsampled data. In this thesis we showed that specialized neural networks perform best for both pure registration as well as usage within an motion-compensated reconstruction pipeline while being fast and efficient. In future work we plan to extend the networks to 3D and test the performance on more diverse datasets.

\chapter*{Kurzfassung}
\begin{otherlanguage}{ngerman}
Magnetresonanztomographie ist weit verbreitet im medizinischen Feld aufgrund vieler Vorteile, allerdings sind die langen Aufnahmezeiten problematisch. Um dies zu bekämpfen werden die Rohdaten, welche vom Scanner aufgenommen werden, unterabgetastet, um den Prozess zu beschleunigen. Dies führt allerdings zu Artefakten in den rekonstruierten Bildern proportional zu der Menge an Unterabtastung. Lange Aufnahmezeiten sind außerdem belastend für den Patienten, sodass es zu Bewegung kommen kann. Besonders physiologische Bewegung von z. B. Lunge oder Herz kann kaum unterdrückt werden. Um das Ausmaß dieser Bewegung einzuschränken kann die Luft angehalten werden während einzelne Bilder aufgenommen werden. Dies ist allerdings sehr schwierig für schwer kranke und betäubte Patienten. Diese Probleme können mittels Bildregistrierung vermindert werden, wobei traditionell rechenintensive und langsame iterative Algorithmen verwendet wurden. In dieser Arbeit wird die Nutzung spezialisierter Neuronaler Netwerke für effiziente und schnelle Bildregistrierung von unterabgetasteten Magnetresonanztomographie-Daten untersucht. Diese Forschungsfrage wurde aus mehreren Blickwinkeln betrachtet und untersucht die Registrierungsqualität sowie Nutzbarkeit in einer bewegungskompensierten Rekonstruktions-Pipeline. Es werden die Grundlagen von Magnetresonanztomographie, Deep Learning und Bildregistrierung erklärt und mehrere Experimente beschrieben. Diese beinhalten diverse Parameter-Tests und Ablationsstudien sowie Downstream-Tests auf zwei verschiedenen Datensätzen, um optimale Netzwerk-Parameter zu finden und die Anwendbarkeit der Netzwerke zu testen. Die Registrierungsqualität wird mit traditionellen Registierungsalgorithmen und aktuellen Neuronalen Netzwerken verglichen,  wobei Segmentierungen für eine akkurate Beurteilung über verschiedene Unterabtastungsraten ermöglichen. In dieser Arbeit konnten wir zeigen, dass spezialisierte Neuronale Netzwerke am besten für pure Registrierung und als Teil einer bewegungskompensierten Rekonstruktions-Pipeline funktionieren, während sie gleichzeitig sehr schnell und effizient sind. In der Zukunft planen wir die Netzwerke auf 3D zu erweitern und diese auf diversen weiteren Datensätzen zu testen.
\end{otherlanguage}