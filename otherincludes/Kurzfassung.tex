\chapter*{Abstract}
While magnetic resonance imaging is a widely used medical imaging technology with many benefits, it suffers from long acquisition times that hamper its effectiveness. 
%This is not only uncomfortable for the patients, but also hinders further image processing as 
To combat this, the k-space data acquired from the scanner is subsampled to speed up the process. 
%When reconstructing images from the raw k-space, subsampling 
This, however, leads to artifacts in the reconstructed images proportionate to the amount of subsampling.
% space limiting the amount of acceleration that can be reasonably achieved. 
Long acquisition times also lead to potential movement between image frames as motion from e.g. the lung and heart that cannot be suppressed leading to further motion artifacts. These problem can be addressed by image registration which was traditionally done by computationally expensive and slow iterative algorithms. In this thesis, the usage of specialized neural networks for efficient and fast image registration of subsampled magnetic resonance imaging data is explored. The research question is approached from multiple different angles examining registration performance as well as usability in a larger motion-compensated reconstruction pipeline. The foundations of magnetic resonance imaging, deep learning and image registration are explained and multiple experiments described. These included diverse parameter tests and ablation studies as well as downstream tests on two different datasets to find optimal model parameters and test the applicability of the networks. The registration performance is compared to traditional registration algorithms and state-of-the-art neural networks using segmentations for accurate assessment across four different amounts of subsampling. We were able to outperform both traditional registration algorithms and state-of-the-art neural networks in terms of Dice score computed on the segmentations even for strongly subsampled data. In this thesis we showed that specialized neural networks perform best for both pure registration as well as usage within an motion-compensated reconstruction pipeline while being fast and efficient. In future work we plan to extend the networks to 3D and test the performance on more diverse datasets.

\chapter*{Kurzfassung}
\begin{otherlanguage}{ngerman}
Obwohl Magnetresonanztomographie eine weit verbreitete medizinische Bildgebungstechnologie mit vielen Vorteilen ist, hat es Probleme mit langen Aufnahmezeiten. Um dies zu bekämpfen werden die k-Raum Daten, welche vom Scanner aufgenommen werden, unterabgetastet um den Prozess zu beschleunigen. Dies führt allerdings zu Artefakten in den rekonstruierten Bilder proportional zu der Menge an Unterabtastung. Lange Aufnahmezeiten führen außerdem auch zu mehr potentieller Bewegung zwischen aufeinanderfolgenden Bildern durch z.B. Lunge und Herz, die nicht unterdrückt werden können, wodurch Bewegungsartefakte entstehen. Diese Probleme können mittels Bildregistrierung vermindet werden wobei traditionell rechenintensive und langsame iterative algorithmen verwendet wurden. In dieser Arbeit wird die Nutzung spezialisierter Neuraler Netwerke für effiziente und schnelle Bild-Registrierung von unterabgetasteter Magnetresonanztomographie Daten untersucht. Diese Forschungsfrage wurde aus mehreren Blickwinkeln betrachtet und untersucht die Registrierungsqualität sowie Nutzbarkeit in einer Bewegungskompensierten Rekonstruktions-Pipeline. Die Grundlagen von Magnetresonanztomographie, Deep Learning und Bild-Registrierung werden erklärt und mehrere Experimente beschrieben. Diese beinhalten diverse Parameter Tests und Ablationsstudien sowie Downstream Tests auf zwei verschiedenen Datensätzen um optimale Netzwerk Parameter zu finden und die Anwendbarkeit der Netzwerke zu testen. Die Registrierungsqualität wird mit traditionellen Registierungsalgorithmen und aktuellen Neuralen Netzwerken verglichen  wobei Segmentierungen for eine akkurate Beurteilung über verschiedene Unterabtastungsraten ermöglicht. In dieser Arbeit konnten wir zeigen, dass spezialisierte Neurale Netzwerke am besten funktionieren für pure Registrierung und als Teil einer Bewegungskompensierten Rekonstruktions-Pipeline, während sie gleichzeitig noch sehr schnell und effizient sind. In der Zukunft planen wir die Netzwerke auf 3D zu erweitern und diese auf diversen weiteren Datensätzen zu testen.
\end{otherlanguage}